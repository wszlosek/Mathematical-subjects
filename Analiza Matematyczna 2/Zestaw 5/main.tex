\documentclass{article}
\usepackage[a4paper, margin={1in, 1in}]{geometry}
\usepackage[utf8]{inputenc}
\usepackage{polski}
\usepackage{mathtools}
\usepackage{amsfonts}
\usepackage{amssymb}
\usepackage{amsmath}
\usepackage{multicol}
\usepackage{paralist}
\usepackage{tabto}
\usepackage{graphicx}
\usepackage{etoolbox}
\usepackage{changepage}
\usepackage{tasks}
\usepackage{pgfplots}
\usepackage{fancyhdr}

\DeclareMathOperator{\arctg}{arctg}
\DeclareMathOperator{\sh}{sh}
\DeclareMathOperator{\ch}{ch}
\DeclareMathOperator{\sgn}{sgn}

\let\arctan\relax
\DeclareMathOperator{\arctan}{arctg}
\let\tan\relax
\DeclareMathOperator{\tan}{tg}

% Dodatkowe deklaracje:

% Koniec dodatkowych deklaracji

\pagestyle{fancy}

\lhead{Analiza Matematyczna 2; 2020/2021}
% Tutaj proszę uzupełnić imię i nazwisko
\rhead{Wojciech Szlosek}

\begin{document}

% Początek każdego zadania należy oznaczać w następujący sposób: 
% \section*{Zadanie [nr_zadania]}
% lub
% \section*{Zadanie [nr_zadania][podpunkt]}


\section*{Zadanie 1b}

$$f(x,y,z) = \frac{x}{x^2 + y^2 + z^2}$$
To dziedziny należą wszystkie $x,y,z$, poza przypadkiem, gdzie $x^2+y^2+z^2=0$. \\ \\
Idea: oblicz pochodną danej zmiennej (wskazuje ją postać symboliczna mianownika), a pozostałe literki traktuj jako stałe. Obliczamy:
$$\frac{\partial f}{\partial x} = \frac{x^2+y^2+z^2 - x \cdot 2x}{(x^2+y^2+z^2)^2} = \frac{-x^2+y^2+z^2}{(x^2+y^2+z^2)^2}$$
$$\frac{\partial f}{\partial y} = \frac{0 - x \cdot 2y}{(x^2+y^2+z^2)^2} = \frac{-2xy}{(x^2+y^2+z^2)^2}$$
$$\frac{\partial f}{\partial z} = \frac{0 - x \cdot 2z}{(x^2+y^2+z^2)^2} = \frac{-2xz}{(x^2+y^2+z^2)^2}$$

\section*{Zadanie 2a}
$$f(x)=\bigg\{ \begin{array}{rl}
x^2 + y^2 & \textrm{gdy $xy = 0$} \\ 1 & \textrm{gdy $xy \neq 0$} \end{array} -- (x_0,y_0) = (0,0)$$
 
$$\frac{\partial f}{\partial x} (0,0) = \lim_{\Delta x \to 0} \frac{f(\Delta x,0) - f(0,0)}{\Delta x} = \lim_{\Delta x \to 0} \frac{(\Delta x)^2}{\Delta x} = 0$$
$$\frac{\partial f}{\partial y} (0,0) = \lim_{\Delta y \to 0} \frac{f(0,\Delta y) - f(0,0)}{\Delta y} = \lim_{\Delta y \to 0} \frac{(\Delta y)^2}{\Delta y} = 0$$

\section*{Zadanie 3c}
$$f(x,y,z) = \frac{1}{\sqrt{x^2+y^2+z^2}}$$
Dziedzina oczywiście obejmuje wszystkie liczby rzeczywiste $x,y,z$ poza przypadkiem, gdzie $x^2 + y^2 +z^2 = 0$. Obliczamy: \\
$$\frac{\partial f}{\partial x} = \frac{0-(\sqrt{x^2+y^2+z^2)^{'}})}{x^2+y^2+z^2}= \frac{\frac{-2x}{2\sqrt{x^2+y^2+z^2}}}{x^2+y^2+z^2} = \frac{-x}{(x^2+y^2+z^2)\sqrt{x^2+y^2+z^2}}$$
Obliczając całkowicie analogicznie, mamy:
$$\frac{\partial f}{\partial y} = \frac{-y}{(x^2+y^2+z^2)\sqrt{x^2+y^2+z^2}}$$
$$\frac{\partial f}{\partial z} = \frac{-z}{(x^2+y^2+z^2)\sqrt{x^2+y^2+z^2}}$$
Następnie:
$$\frac{\partial^{2}f}{\partial x^2} = \frac{\partial}{\partial x} (\frac{\partial f}{\partial x}) = \frac{\partial}{\partial x}(\frac{-x}{(x^2+y^2+z^2)\sqrt{x^2+y^2+z^2}}) = \frac{-(x^2+y^2+z^2)\sqrt{x^2+y^2+z^2}+3x^2 \sqrt{x^2+y^2+z^2}}{[(x^2+y^2+z^2)\sqrt{x^2+y^2+z^2}]^2} = $$
$$= - \frac{-2x^2+y^2+z^2}{(x^2+y^2+z^2)^{\frac{5}{2}}}$$

$$\frac{\partial^{2}f}{\partial x \partial y} = \frac{\partial}{\partial x}(\frac{\partial f}{\partial y}) = \frac{\partial}{\partial x} (\frac{-y}{(x^2+y^2+z^2)\sqrt{x^2+y^2+z^2}}) = -y \cdot \frac{\partial}{\partial x} (((x^2+y^2+z^2)\sqrt{x^2+y^2+z^2}))^{-1}) = $$
$$= y \cdot \frac{1}{[(x^2+y^2+z^2)\sqrt{x^2+y^2+z^2}]^2}\cdot \frac{\partial}{\partial x} ((x^2+y^2+z^2)\sqrt{x^2+y^2+z^2}) = $$ $$\frac{y}{[(x^2+y^2+z^2)\sqrt{x^2+y^2+z^2}]^2}\cdot 3x\sqrt{x^2+y^2+z^2} = \frac{3xy}{(x^2+y^2+z^2)^2\sqrt{x^2+y^2+z^2}}$$ \\
Mając za sobą toporne obliczenia, następne zależności możemy ustalać już całkiem analogicznie do poprzednich. I tak, mamy:
$$\frac{\partial^{2}f}{\partial x \partial z} = \frac{\partial}{\partial x}(\frac{-z}{(x^2+y^2+z^2)\sqrt{x^2+y^2+z^2}}) = ... = \frac{3xz}{(x^2+y^2+z^2)^2 \sqrt{x^2+y^2+z^2}}$$
$$\frac{\partial^{2}f}{\partial y \partial x} = \frac{\partial}{\partial y}(\frac{-x}{(x^2+y^2+z^2)\sqrt{x^2+y^2+z^2}}) = ... = \frac{3xy}{(x^2+y^2+z^2)^2\sqrt{x^2+y^2+z^2}}$$

$$\frac{\partial^{2}f}{\partial y^2} = \frac{\partial}{\partial y}(\frac{-y}{(x^2+y^2+z^2)\sqrt{x^2+y^2+z^2}}) = ... = - \frac{-2yz+x^2+z^2}{(x^2+y^2+z^2)\sqrt{x^2+y^2+z^2}}$$

$$\frac{\partial^{2}f}{\partial y \partial z} = \frac{\partial}{\partial y}(\frac{-z}{(x^2+y^2+z^2)\sqrt{x^2+y^2+z^2}}) = ... = \frac{3zy}{(x^2+y^2+z^2)^2\sqrt{x^2+y^2+z^2}}$$

$$\frac{\partial^{2}f}{\partial z \partial x} = \frac{\partial}{\partial z}(\frac{-x}{(x^2+y^2+z^2)\sqrt{x^2+y^2+z^2}}) = ... = \frac{3xz}{(x^2+y^2+z^2)^2\sqrt{x^2+y^2+z^2}}$$

$$\frac{\partial^{2}f}{\partial z \partial y} = \frac{\partial}{\partial z}(\frac{-y}{(x^2+y^2+z^2)\sqrt{x^2+y^2+z^2}}) = ... = \frac{3yz}{(x^2+y^2+z^2)^2\sqrt{x^2+y^2+z^2}}$$

$$\frac{\partial f}{\partial z^2} = \frac{\partial}{\partial z}(\frac{-z}{(x^2+y^2+z^2)\sqrt{x^2+y^2+z^2}}) = ... = - \frac{-2z^2+x^2+y^2}{(x^2+y^2+z^2)\sqrt{x^2+y^2+z^2}}$$ \\ \\
Odp.: Zatem ostatecznie, jak widać, mamy: $\frac{\partial^2 f}{\parial x \partial y}=\frac{\partial^2 f}{\parial y \partial x}$; $\frac{\partial^2 f}{\parial x \partial z} = \frac{\partial^2 f}{\parial z \partial x}$; $\frac{\partial^2 f}{\parial y \partial z} = \frac{\partial^2 f}{\parial z \partial y}$. 

\end{document}