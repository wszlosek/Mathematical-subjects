\documentclass{article}
\usepackage[a4paper, margin={1in, 1in}]{geometry}
\usepackage[utf8]{inputenc}
\usepackage{polski}
\usepackage{mathtools}
\usepackage{amsfonts}
\usepackage{amssymb}
\usepackage{amsmath}
\usepackage{multicol}
\usepackage{paralist}
\usepackage{tabto}
\usepackage{graphicx}
\usepackage{etoolbox}
\usepackage{changepage}
\usepackage{tasks}
\usepackage{pgfplots}
\usepackage{fancyhdr}

\DeclareMathOperator{\arctg}{arctg}
\DeclareMathOperator{\tg}{tg}
\DeclareMathOperator{\sh}{sh}
\DeclareMathOperator{\ch}{ch}
\DeclareMathOperator{\sgn}{sgn}

\let\arctan\relax
\DeclareMathOperator{\arctan}{arctg}
\let\tan\relax
\DeclareMathOperator{\tan}{tg}

% Dodatkowe deklaracje:

% Koniec dodatkowych deklaracji

\pagestyle{fancy}

\lhead{Analiza Matematyczna 2; 2020/2021}
% Tutaj proszę uzupełnić imię i nazwisko
\rhead{Wojciech Szlosek}

\begin{document}

\section*{Zadanie 2b}

$$\int \int \int_U z \ln(x^y y^x) \ dx dy dz \ \ \ \ U = [1,e] x[1,e]x[0,1]$$ \\ 
Korzystając, z tego, że: $\ln(x^y y^x) = \ln(x^y)+\ln(y^x) = y\ln(x) + x \ln(y)$, mamy: \\
$$\int \int \int_U z \ln(x^y y^x) \ dx dy dz = \int \int \int_U z (y\ln(x) + x\ln(y)) = \int \int \int_U zy\ln(x) + \int \int \int_U zx \ln(y) = $$
$$= \Big( \int_0^1 z \ dz \Big) \cdot \Big( \int_1^e y \ dy \Big) \cdot \Big( \int_1^e \ln(x) \ dx \Big) +  \Big( \int_0^1 z \ dz \Big) \cdot \Big( \int_1^e \ln(y) \ dy \Big) \cdot \Big( \int_1^e x \ dx \Big)$$

\section*{Zadanie 3c}

$$z = x^2+y^2 \ \ \ \ z = \sqrt{20-x^2-y^2}$$ \\ 
Równania z treści przedstawiają dwie (zwrócone przeciwnie) paraboloidy, nasz obszar U będzie przestrzenią nad $z = x^2 + y^2$ oraz pod paraboloidą $z = \sqrt{20-x^2-y^2}$. Niech rzut obszaru $U$ na płazszczyznę będzie oznaczony przez $D$ - ma on postać koła wyznaczonego przez przecięcie obu paraboloid, a więc wystarczy rozwiązać równanie $x^2 + y^2 = \sqrt{20-x^2 -y^2}$. Rozważmy pomocnicze zmienne a = $x^2$, b = $y^2$. Mamy po podniesieniu obustronnie do kwadratu:
$$x^4 +2x^2y^2 + y^4 = 20 - x^2 - y^2$$
$$x^4 + y^4 + 2x^2y^2 + x^2 + y^2 - 20 = 0$$
$$a^2 + b^2 +2ab + a + b - 20 = 0$$
$$a^2 + a(2b + 1) + b^2 +b - 20 = 0$$
Rozwiązując to równanie kwadratowe (np. za pomocą delty), mamy dwa rozwiązania:
$$a_1 = 4 - b \ \ \ \ \ a_2 = -5 - b$$
A zatem: $x^2 = 4 - y^2$ lub $x^2 = -5 - y^2$
Pamiętając, że $x^2, y^2 \geq 0$, eliminujemy jeden przypadek i dostajemy wniosek: $x^2 + y^2 = 4$. Zatem nrozważany rzut obszaru $D$ jest wyznaczony poprzez nierówność koła: $x^2 + y^2 \leq 4$ - koło o środku w punkcie (0,0) i promieniu równym 2. Obszar normalny względem osi OX: $-2 \leq x \leq 2$, $-\sqrt{4-x^2} \leq y \leq \sqrt{4-x^2}$ oraz rozważenie ograniczeń $z$ (z treści zadania): $x^2 + y^2 \leq z \leq \sqrt{20-x^2-y^2}$. \\ \\
Ostatecznie otrzymujemy więc odpowiedź do zadania:
$$\int_{-2}^{2} \ dx \int_{-\sqrt{4-x^2}}^{\sqrt{4-x^2}} \ dy \int_{x^2+y^2}^{\sqrt{20-x^2-y^2}} \ f(x,y,z) \ dz$$


\section*{Zadanie 4a}

Od razu widzimy i możemy rozpisać określenie obszaru całkowania $U$, mamy 'warunki':
$$0 \leq x \leq 1 \ \ \ \ \ 0 \leq y \leq 2-2x \ \ \ \ \ 0 \leq z \leq 3-3x-\frac{3}{2}y$$
Zauważmy, że w naszym przykładzie mamy 6 możliwości co do kolejności całkowania (wyczerpujemy wszystjie możliwości), tj.:
$$I.  \ \ \ 0 \leq x \leq 1 \ \ \ \ 0 \leq y \leq 2 - 2x \ \ \ \ 0 \leq z \leq 3 - 3x - \frac{3}{2}y$$
$$II.  \ \ \ 0 \leq x \leq 1 \ \ \ \ 0 \leq z \leq 3 - 3x \ \ \ \ 0 \leq y \leq 2 - 2x - \frac{2}{3}y$$
$$III.  \ \ \ 0 \leq y \leq 2 \ \ \ \ 0 \leq z \leq 3 - \frac{3}{2}y \ \ \ \ 0 \leq x \leq 1 - \frac{1}{3}z - \frac{1}{2}y$$
$$IV.  \ \ \ 0 \leq y \leq 2 \ \ \ \ 0 \leq x \leq 1 - \frac{1}{2}y \ \ \ \ 0 \leq z \leq 3 - 3x - \frac{3}{2}y$$
$$V.  \ \ \ 0 \leq z \leq 3 \ \ \ \ 0 \leq x \leq 1 - \frac{1}{3}z \ \ \ \ 0 \leq y \leq 2 - 2x - \frac{2}{3}z$$
$$VI.  \ \ \ 0 \leq z \leq 3 \ \ \ \ 0 \leq y \leq 2 - \frac{2}{3}z \ \ \ \ 0 \leq x \leq 1 - \frac{1}{3}z - \frac{1}{2}y$$ \\
Skąd wzięły się takie przypadki? Przedstawmy na podstawie ostatniego przypadku: rozważamy $0 \leq z \leq 3 - \frac{3}{2}y$, z tego mamy $0 \leq \frac{3}{2}y \leq 3-z $, a więc ostatecznie $0 \leq y \leq 2 - \frac{2}{3}z$ - co wstawiamy do naszych warunków w danym przypadku. Resztę możliwości rozważamy całkowicie analogicznie. \\ \\
Zauważmy ponadto, że przypadek $I.$ to przypadek rozpatrywany w treści zadania. Odpowiedzią do zadania będzie zatem sześć całek, które wypiszę w powyższej kolejności rozważanych przypadków:
$$\int_0 ^ {1} \ dx \int_0 ^ {2-2x} \ dy \int_0 ^{3-3x-\frac{3}{2}y} f(x,y,z) \ dz$$
$$\int_0 ^ {1} \ dx \int_0 ^ {3-3x} \ dz \int_0 ^{2-2x-\frac{2}{3}z} f(x,y,z) \ dy$$
$$\int_0 ^ {2} \ dy \int_0 ^ {3-\frac{3}{2}y} \ dz \int_0 ^{1 - \frac{1}{3}z - \frac{1}{2}y} f(x,y,z) \ dx$$
$$\int_0 ^ {2} \ dy \int_0 ^ {1 - \frac{1}{2}y} \ dx \int_0 ^{3-3x-\frac{3}{2}y} f(x,y,z) \ dz$$
$$\int_0 ^ {3} \ dz \int_0 ^ {1 - \frac{1}{3}z} \ dx \int_0 ^{2-2x-\frac{2}{3}y} f(x,y,z) \ dy$$
$$\int_0 ^ {3} \ dz \int_0 ^ {2 - \frac{2}{3}z} \ dy \int_0 ^{1-\frac{1}{3}z-\frac{1}{2}y} f(x,y,z) \ dx$$
\end{document}
