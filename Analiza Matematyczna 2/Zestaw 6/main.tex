\documentclass{article}
\usepackage[a4paper, margin={1in, 1in}]{geometry}
\usepackage[utf8]{inputenc}
\usepackage{polski}
\usepackage{mathtools}
\usepackage{amsfonts}
\usepackage{amssymb}
\usepackage{amsmath}
\usepackage{multicol}
\usepackage{paralist}
\usepackage{tabto}
\usepackage{graphicx}
\usepackage{etoolbox}
\usepackage{changepage}
\usepackage{tasks}
\usepackage{pgfplots}
\usepackage{fancyhdr}

\DeclareMathOperator{\arctg}{arctg}
\DeclareMathOperator{\sh}{sh}
\DeclareMathOperator{\ch}{ch}
\DeclareMathOperator{\sgn}{sgn}

\let\arctan\relax
\DeclareMathOperator{\arctan}{arctg}
\let\tan\relax
\DeclareMathOperator{\tan}{tg}

% Dodatkowe deklaracje:

% Koniec dodatkowych deklaracji

\pagestyle{fancy}

\lhead{Analiza Matematyczna 2; 2020/2021}
% Tutaj proszę uzupełnić imię i nazwisko
\rhead{Wojciech Szlosek}

\begin{document}

% Początek każdego zadania należy oznaczać w następujący sposób: 
% \section*{Zadanie [nr_zadania]}
% lub
% \section*{Zadanie [nr_zadania][podpunkt]}


\section*{Zadanie 1b}

$$f(x,y)=\bigg\{\begin{array}{rl}
(x^2+y^2) \sin{\frac{1}{x^2+y^2}} & \textrm{dla $(x,y) \neq (0,0)$} \\ 0 & \textrm{dla $(x,y) = (0,0)$} \end{array} (x_0,y_0) = (0,0) $$

Obliczamy z definicji:
$$\frac{\partial f}{\partial x} (0,0) = \lim_{h \to 0} \frac{f(0+h,0)-f(0,0)}{h} = \lim_{h \to 0} \frac{0}{h} = 0$$
$$\frac{\partial f}{\partial y} (0,0) = \lim_{h \to 0} \frac{f(0,0+h)-f(0,0)}{h} = 0$$ \\
Oznaczmy: niech $w = \Delta x, k = \Delta y$. Zgodnie z podręcznikiem, sprawdźmy czy podana niżej granica będzie równa 0, ten warunek pozwoli nam stwierdzić, że funkcja jest różniczkowalna.
$$\lim_{(w,k) \to (0,0)} \frac{f(x_0+w,y_0+k)-f(x_0,y_0) - \frac{\partial f}{\partial x}(x_0,y_0)w - \frac{\partial f}{\partial y} (x_0,y_0)k}{\sqrt{w^2 + k^2}} = \lim_{(w,k) \to (0,0)} \frac{(w^2+k^2) \sin{\frac{1}{w^2+k^2}}}{\sqrt{w^2+k^2}}$$

Zajmijmy się oszacowaniem z modułem, mianowicie zauważmy:
$$\Big| \frac{(w^2+k^2) \sin{\frac{1}{w^2+k^2}}}{\sqrt{w^2+k^2}}\Big| \leq \frac{w^2+k^2}{\sqrt{w^2+k^2}} = \sqrt{w^2+k^2} \to 0$$
Skoro wyeażenie po prawej stronie dąży do zzera, to wyrażenie w module również dąży do 0 przy oszacowaniu sinusa przez 1. \\ \\
Zatem granica wynosi 0, a więc funkcja jest różniczkowalna. (odp.)

\section*{Zadanie 3b}

Przyjmijmy: $f(x,y,z) = \sqrt[3]{x^3+y^3+z^3}$
$$\frac{\partial f}{\partial x} = \frac{1}{3\sqrt[3]{(x^3+y^3+z^3)^2}} \cdot (x^3+y^3+z^3)^{'}_{x} = \frac{x^2}{\sqrt[3]{(x^3+y^3+z^3)^2}}$$
Analogicznie:
$$\frac{\partial f}{\partial y} = \frac{y^2}{\sqrt[3]{(x^3+y^3+z^3)^2}}$$
$$\frac{\partial f}{\partial z} = \frac{z^2}{\sqrt[3]{(x^3+y^3+z^3)^2}}$$ \\
$$\Delta f = \frac{\partial f}{\partial x} \Delta x + \frac{\partial f}{\partial y} \Delta y + \frac{\partial f}{\partial z} \Delta z$$
Niech $x = 3, y = 4, z = 5; \Delta x = -0.07, \Delta y = 0.05, \Delta z = -0.01$. Mamy wtedy:
$$f(3,4,5) = \sqrt[3]{3^3+4^3+5^3} = 6$$
$$\frac{\partial f}{\partial x} (3,4,5) = \frac{3^2}{\sqrt[3]{(3^3+4^3+5^3)^2}}$$
Podobnie $\frac{\partial f}{\partial y} (3,4,5) = \frac{16}{36}$, $\frac{\partial f}{\partial z} (3,4,5) = \frac{25}{36}$.
$$\Delta f = \frac{9}{36} * (-0.07) + \frac{16}{36} \cdot 0.05 + \frac{25}{36} \cdot (-0.01) = \frac{-1}{450}$$ \\
Ostatecznie mamy:
$$\sqrt[3]{(2.93)^3 +(4.05)^3 + (4.99)^3} = f(3,4,5) - \Delta f = 6 - \frac{1}{450} = 5\frac{449}{450}$$

\section*{Zadanie 5b}

$$z = f(u,v,w) = \arcsin{\frac{u}{v+w}} (:u = e^{\frac{x}{y}}, v = x^2+y^2, w = 2xy)$$
Obliczamy:
$$z^{'}_{x} = \frac{1}{\sqrt{1-(\frac{u}{
v+w})^2}} \cdot \frac{u^{'}(v+w)-u(v+w)^{'}}{(v+w)^2}$$
$$z^{'}_{y} = \frac{1}{\sqrt{1-(\frac{u}{v+w})^2}} \cdot \frac{u^{'}(v+w)-u(v+w)^{'}}{(v+w)^2}$$
$$u^{'}_{x} = \frac{e^{\frac{x}{y}}}{y}; u^{'}_{y} = \frac{-xe^{\frac{x}{y}}}{y^2}; (v+w)^{'}_x = 2x+2y = (v+w)^{'}_y$$ \\
A zatem (po drodze wykorzystamy m.in. wzór na pochodną iloczynu):
$$z^{'}_{x} = \frac{1}{\sqrt{1-\frac{e^{\frac{2x}{y}}}{(x+y)^4}}} \cdot \frac{ \frac{e^{\frac{x}{y}}}{y} (x+y)^2 - \frac{y \cdot e^{\frac{x}{y}} (2x+2y)}{y}}{(x+y)^4} = \frac{1}{\sqrt{1-\frac{e^{\frac{2x}{y}}}{(x+y)^4}}} \cdot \frac{e^{\frac{x}{y}} (x^2+2xy+y^2-2xy-2y^2)}{y(x+4)^4} = $$
$$= \frac{e^{\frac{x}{y}} (x+y)(x-y)}{\sqrt{y^2(x+y) \cdot (1 - \frac{e^{\frac{2x}{y}}}{(x+y)^4})}} = \frac{e^{\frac{x}{y}} (x+y)(x-y)}{y(x+y)^2 \cdot \sqrt{(x+y)^4 - e^{\frac{2x}{y}}}} = \frac{e^{\frac{x}{y}}(x-y)}{y(x+y) \cdot \sqrt{(x+y)^4 - e^{\frac{2x}{y}}}}$$

$$z^{'}_{y} = \frac{1}{\sqrt{1-\frac{e^{\frac{2x}{y}}}{(x+y)^4}}} \cdot \frac{ \frac{-e^{\frac{x}{y}}}{y^2} (x+y)^2 - \frac{y^2 (2x+2y) \cdot e^{\frac{x}{y}}}{y^2} }{(x+y)^4}
= \frac{1}{\sqrt{1-\frac{e^{\frac{2x}{y}}}{(x+y)^4}}} \cdot \frac{e^{\frac{x}{y}} \cdot (-x^2-2xy-y^2+2xy^2+2y^3)}{y^2(x+y)^4} = $$ 
$$= \frac{-e^{\frac{x}{y}} (x^2+2xy+y^2-2xy^2-2y^3)}{\sqrt{y^4 (x+y)^8 [1-\frac{e^{\frac{2x}{y}}}{(x+y)^4}]}} = \frac{-e^{\frac{x}{y}} (x^2+2xy+y^2-2xy^2-2y^3)}{y^2 (x+y)^2 \sqrt{(x+y)^4 - e^{\frac{2x}{y}}}} = \frac{-e^{\frac{x}{y}}(x+y-2y^2)}{y^2 (x+y) \sqrt{(x+y)^4 - e^{\frac{2x}{y}}}}$$


\end{document}