\documentclass{article}
\usepackage[a4paper, margin={1in, 1in}]{geometry}
\usepackage[utf8]{inputenc}
\usepackage{polski}
\usepackage{mathtools}
\usepackage{amsfonts}
\usepackage{amssymb}
\usepackage{amsmath}
\usepackage{multicol}
\usepackage{paralist}
\usepackage{tabto}
\usepackage{graphicx}
\usepackage{etoolbox}
\usepackage{changepage}
\usepackage{tasks}
\usepackage{pgfplots}
\usepackage{fancyhdr}

\DeclareMathOperator{\arctg}{arctg}
\DeclareMathOperator{\tg}{tg}
\DeclareMathOperator{\sh}{sh}
\DeclareMathOperator{\ch}{ch}
\DeclareMathOperator{\sgn}{sgn}

\let\arctan\relax
\DeclareMathOperator{\arctan}{arctg}
\let\tan\relax
\DeclareMathOperator{\tan}{tg}

% Dodatkowe deklaracje:

% Koniec dodatkowych deklaracji

\pagestyle{fancy}

\lhead{Analiza Matematyczna 2; 2020/2021}
% Tutaj proszę uzupełnić imię i nazwisko
\rhead{Wojciech Szlosek}

\begin{document}

\section*{Zadanie 1c}

Mamy spółczynniki kierunkowe prostych - $k: (a1, a2, a3) = (1,1,0)$ x $(0,1,1) = (1,-1,0)$, $l: (b1,b2,b3) = (1,-1,0)$ x $(0,0,1) = (-1,-1,0)$. Zauważmy, że np. punkt $P(0,1,-1)$ należy do prostek $k$, a $R(0,3,2)$ do prostej $l$. Wyznaczmy równania kierunkowe prostych:
$$k: \frac{x-0}{1} = \frac{y-1}{-1} = \frac{z+1}{0} \ \ \ \ l: \frac{x-0}{-1} = \frac{y-3}{-1} = \frac{z-2}{0}$$
Wektor normalny płaszczyzn: $(1,-1,0)$ x $(-1,-1,0)$ = $(0,0,-2)$
Równania płaszczyzn zawierające dane proste:
$$\pi_{k} = 0(x-0)+0(y-1)-2(z+1) = 0 \implies z = -1$$
$$\pi_{l} = 0(x-0)+0(y-3)-2(z-2) = 0 \implies z = 2$$
Stąd odległość wynosi:
$$d = \frac{|-2-4|}{\sqrt{0^2+0^2+(-2)^2}} = 3$$

\section*{Zadanie 1d}

Dziedzina ($D$): $a,b,c > 0$. Wiemy, że $V = abc = 216$, stąd $c = \frac{V}{ab} = \frac{216}{ab}$. Koszt budowy określmy pewną funkcją $f$.
$$f = 30 \cdot 2(a+b)\cdot c + 40ab + 20ab = 60V \cdot \frac{a+b}{ab}+60ab$$
$$f^{'}_{a} = \frac{-60V}{a^2}+60b \ \ \ \ f^{'}_{b} = \frac{-60V}{b^2}+60a$$
Przyrównując pochodne przy odpowiednich zmiennych do zera, mamy:
$$\frac{-60V}{a^2}+60b = 0 \implies V = a^2b$$
$$\frac{-60V}{b^2}+60a = 0 \implies V = ab^2$$
Stąd mamy, że $a = b = 216^{\frac{1}{3}} = 6$ m. \\
Obliczając drugie pochodne, a następnie podkładając te do wiadomego wyznacznika, mamy, że $W(6,6) = 120 \cdot 120 - (60 \cdot 60) > 0$. A zatem rzeczywiście jest tam ekstremum (powiem więcej: minimum lokalne) Pora na krótkie uzasadnienie globalności. Zbadajmy zachowanie się funkcji kiedyż punkt $(x,y) \in D$ dąży do brzegu obszaru D. Mamy więc dwie (analogiczne) granice, które (jak się okazuje, podstawiając do nich nasze dane) dążą do nieskończoności. Stąd uzasadnienie globalności. \\ \\
Ze wzoru na $c$, mamy, że również i $c = 6$ m. \\
Odpowiedź: a = b = c = 6 m.


\section*{Zadanie 2a}
$$f(x,y) = x^3+x-y^3-y = 0, \ \ \ \ (2,2)$$
Równanie stycznej do krzywej w punkcie $(x_0,y_0)$ ma postać: 
$$y-y_0 = - \frac{\frac{\partial f}{\partial x}(x_0,y_0)}{\frac{\partial f}{\partial y}(x_0,y_0)}(x-x_0)$$
Obliczamy:
$$\frac{\partial f}{\partial x} = 3x^2+1 \ \ \ \ \frac{\partial f}{\partial x} (2,2) = 13$$
$$\frac{\partial f}{\partial y} = -3y^2-1 \ \ \ \ \frac{\partial f}{\partial y} (2,2) = -13$$
Zatem, podstawiając do wzoru, mamy odpowiedź, że równanie ma postać:
$$y-2 = - \frac{13}{-13}(x-2) \implies y = x$$

\end{document}
