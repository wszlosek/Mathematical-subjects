\documentclass{article}
\usepackage[a4paper, margin={1in, 1in}]{geometry}
\usepackage[utf8]{inputenc}
\usepackage{polski}
\usepackage{mathtools}
\usepackage{amsfonts}
\usepackage{amssymb}
\usepackage{amsmath}
\usepackage{multicol}
\usepackage{paralist}
\usepackage{tabto}
\usepackage{graphicx}
\usepackage{etoolbox}
\usepackage{changepage}
\usepackage{tasks}
\usepackage{pgfplots}
\usepackage{fancyhdr}

\DeclareMathOperator{\arctg}{arctg}
\DeclareMathOperator{\sh}{sh}
\DeclareMathOperator{\ch}{ch}
\DeclareMathOperator{\sgn}{sgn}

\let\arctan\relax
\DeclareMathOperator{\arctan}{arctg}
\let\tan\relax
\DeclareMathOperator{\tan}{tg}

\pagestyle{fancy}

\lhead{Analiza Matematyczna 2; 2020/2021}
\rhead{Wojciech Szlosek}

\begin{document}

\section*{Zadanie 1b}

$$\int_{0}^{\infty} 2^{-x} \,dx = \lim_{T\to \infty} \int_{0}^{T} 2^{-x} \,dx = ...$$

Niech $t = -x, dx = -dt$, wówczas po podstawieniu:
$$ ... = \lim_{T\to \infty} - \int_{0}^{T} 2^{t} \,dt = \lim_{T\to \infty} [\frac{-2^t}{\ln2}]_{0}^{T} = \lim_{T\to \infty} [\frac{-2^{-x}}{\ln2}]_{0}^{T} = \lim_{T\to \infty} \frac{-\frac{1}{2^T}+1}{\ln2} = \frac{1}{\ln2}$$

Zatem badana całka jest zbieżna. (odp.)

\section*{Zadanie 2a}

$$\int_{10}^{\infty} \frac{1}{\sqrt{x}-3} \,dx$$

Dla każdego $x \geq 10$, prawdziwym jest, że: $0 \leq \frac{1}{\sqrt{x}} \leq \frac{1}{\sqrt{x}-3}$, ponadto:

$$* = \int_{10}^{\infty} \frac{1}{\sqrt{x}} \,dx = \lim_{T\to \infty} \int_{10}^{T} x^{-\frac{1}{2}} \,dx = \lim_{T\to \infty} [2\sqrt{x}]_{10}^{T} = \infty$$

Zatem całka (*) jest rozbieżna do $\infty$. Z kryterium porównawczego wynika rozbieżność do $\infty$ również badanej całki wejściowej. (odp.)

\section*{Zadanie 5e}

$$\int_{0}^{\pi} \frac{\sin^3{x}}{x^4} \,dx$$

Przyjmując w kryterium ilorazowym zbieżność całek niewłaściwych drugiego rodzaju: $f(x) = \frac{\sin^3{x}}{x^4}$ i $g(x) = \frac{1}{x}$, mamy:

$$k = \lim_{x \to 0^{+}} \frac{f(x)}{g(x)} = \lim_{x \to 0^{+}} \frac{\sin^3{x}}{x^3} = 1$$
$$0 < k < \infty$$

Zbadajmy zbieżność:
$$\int_{0}^{\pi}\frac{1}{x} \,dx = \lim_{T\to 0^{+}} \int_{T}^{\pi} \frac{1}{x} \,dx = \lim_{T\to 0^{+}} [\ln{x}]_{T}^{\pi} = \lim_{T\to 0^{+}} (\ln{\pi}-\ln{T}) = \infty$$

Zatem badana całka wejściowa również jest rozbieżna. (odp.)

\end{document}