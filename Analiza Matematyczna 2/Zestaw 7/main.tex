\documentclass{article}
\usepackage[a4paper, margin={1in, 1in}]{geometry}
\usepackage[utf8]{inputenc}
\usepackage{polski}
\usepackage{mathtools}
\usepackage{amsfonts}
\usepackage{amssymb}
\usepackage{amsmath}
\usepackage{multicol}
\usepackage{paralist}
\usepackage{tabto}
\usepackage{graphicx}
\usepackage{etoolbox}
\usepackage{changepage}
\usepackage{tasks}
\usepackage{pgfplots}
\usepackage{fancyhdr}

\DeclareMathOperator{\arctg}{arctg}
\DeclareMathOperator{\sh}{sh}
\DeclareMathOperator{\ch}{ch}
\DeclareMathOperator{\sgn}{sgn}

\let\arctan\relax
\DeclareMathOperator{\arctan}{arctg}
\let\tan\relax
\DeclareMathOperator{\tan}{tg}

% Dodatkowe deklaracje:

% Koniec dodatkowych deklaracji

\pagestyle{fancy}

\lhead{Analiza Matematyczna 2; 2020/2021}
% Tutaj proszę uzupełnić imię i nazwisko
\rhead{Wojciech Szlosek}

\begin{document}

% Początek każdego zadania należy oznaczać w następujący sposób: 
% \section*{Zadanie [nr_zadania]}
% lub
% \section*{Zadanie [nr_zadania][podpunkt]}


\section*{Zadanie 3a}

$$f(x,y) = \sin(x^2+y^2) \ \ (x_0,y_0) = (0,0) \ \ n = 3$$ \\
Obliczamy. W między-czasie wiele rzeczy liczymy analogicznie do siebie wzajemnie. Następnie otrzymane wyniki wstawimy do wzoru Taylora (podanego np. w podręczniku i zbioru zadań). Mamy:
$$\frac{\partial f}{\partial x}(0,0) = [\cos(x^2+y^2) \cdot 2x]_{(0,0)} = 0$$
$$\frac{\partial f}{\partial y}(0,0) = [\cos(x^2+y^2) \cdot 2y]_{(0,0)} = 0$$

$$\frac{\partial^2 f}{\partial x^2}(0,0) = [\frac{\partial}{\partial x} (2x \cdot \cos(x^2+y^2))]_{(0,0)} = 2$$
$$\frac{\partial^2 f}{\partial y^2}(0,0) = [\frac{\partial}{\partial y} (2y \cdot \cos(x^2+y^2))]_{(0,0)} = 2$$
$$\frac{\partial^2 f}{\partial x \partial y}(0,0) = 0$$
Zatem:
$$f(x,y) = 0 + \frac{1}{1!}[0 \cdot x + 0 \cdot y] + \frac{1}{2!}[2x^2 + 2 \cdot 0 \cdot xy + 2y^2] + R_3(x,y)$$
Niech $(E, N) = (\theta x, \theta y)$, gdzie $0 < \theta < 1$. Dalej mamy:

$$\frac{\partial^3 f}{\partial x^3}(E,N) = -4E[3 \sin(E^2+N^2)+2E^2\cos(E^2+N^2)]$$
$$\frac{\partial^3 f}{\partial x \partial y^2}(E,N) = -4E[\sin(E^2+N^2)+2N^2\cos(E^2+N^2)]$$
$$\frac{\partial^3 f}{\partial x^2 \partial y}(E,N) = -4N[\sin(E^2+N^2)+2E^2\cos(E^2+N^2)]$$
$$\frac{\partial^3 f}{\partial y^3}(E,N) = -4N[3 \sin(E^2+N^2)+2N^2\cos(E^2+N^2)]$$ \\
Zatem ostateczny wynik to:
$$f(x,y) = x^2+y^2+ \frac{1}{3!}\Bigg[(-4E[3 \sin(E^2+N^2)+2E^2\cos(E^2+N^2)])x^3 + (-12N[\sin(E^2+N^2)+2E^2\cos(E^2+N^2)]) x^2y + $$
$$+(-12E[\sin(E^2+N^2)+2N^2\cos(E^2+N^2)])xy^2 + (-4N[3\sin(E^2+N^2)+2N^2\cos(E^2+N^2)])y^3 \Bigg]$$


\section*{Zadanie 4b}

$$f(x,y) = 2x^4 - 3y^7$$ \\
$$\frac{\partial f}{\partial x} = 0 \ \ \ \ 8x^3 = 0 \ \ \ x = 0$$
$$\frac{\partial f}{\partial y} = 0 \ \ \ -21y^6 = 0 \ \ \ y = 0$$
Jednak funkcja nie ma ekstremum lokalnego w punkcie (0,0). Pokażemy, że w otoczeniu tego punktu można znaleźć punkty w których funkcja ma wartość mniejszą of f(0,0) = (0,0) oraz punkty w których ma ona wartość większą of f(0,0) = (0,0). \\
Rzeczywiście, jakiekolwiek byśmy wybrali otoczenie punktu (0,0), to dla dostatecznie dużej liczby naturalnej $n$ punkty $(\frac{1}{n},0)$, $(0,\frac{1}{n})$ będą należały do tego otoczenia. Ponadto mamy:
$$f(\frac{1}{n},0) = \frac{2}{n^4} > 0 = f(0,0) \ \ i \ \ f(0,\frac{1}{n}) = \frac{-3}{n^7} < 0 = f(0,0)$$


\section*{Zadanie 5c}

$$f(x,y) = x^3 + 3xy^2-51x-24y$$ \\
$$\frac{\partial f}{\partial x} = 3x^2 + 3y^2 -51 \ \ \ \ \ \ \ \frac{\partial f}{\partial y} = 6xy - 24 \ \ \ \ \ \ \ \ \ \frac{\partial^2 f}{\partial x^2} = 6x\ \ \ \ \ \ \ \ \ \frac{\partial^2 f}{\partial y^2} = 6x \ \ \ \ \ \ \ \ \frac{\partial^2 f}{\partial x \partial y} = 6y$$
$$\frac{\partial f}{\partial x} = 0, \ \ \frac{\partial f}{\partial y} = 0 \implies (x = 1, y = 4) \ \ v \ \ (x = -1, y = -4) \ \ v \ \ (x = 4, y = 1) \ \ v \ \ (x = -4, y = -1)$$ \\
Wyznacznik macierzy Hessego (wzór poniżej):
$$g(x,y) = \frac{\partial^2 f}{\partial x^2}(x,y) \cdot \frac{\partial^2 f}{\partial y^2}(x,y) - (\frac{\partial^2 f}{\partial x \partial y}(x,y))^2$$
$$g(1,4) = 6 \cdot 6 - 24^{24} < 0$$
Zatem brak ekstremum w tym punkcie. 
$$g(4,1) = 24 \cdot 24 - 6^2 > 0 \ \ i \ \ \frac{\partial^2 f}{\partial x^2}(4,1) = 24 > 0$$
Zatem mamy tutaj minimum lokalne. \\ \\
Analogicznie sprawdzamy, dla pozostałych dwóch punktów.
$$g(-1,-4) < 0 \ ; \ \ \ g(-4,-1) > 0 \ \ i \ \ \frac{\partial^2 f}{\partial x^2}(-4,-1) < 0$$
Zatem w (-1,-4) brak ekstremum, z kolei w drugim rozważanym punkcie maksimum lokalne.

\end{document}