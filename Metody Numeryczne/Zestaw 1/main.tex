\documentclass{article}
\usepackage[a4paper, margin={1in, 1in}]{geometry}
\usepackage[utf8]{inputenc}
\usepackage{polski}
\usepackage{mathtools}
\usepackage{amsfonts}
\usepackage{amssymb}
\usepackage{amsmath}
\usepackage{multicol}
\usepackage{paralist}
\usepackage{tabto}
\usepackage{graphicx}
\usepackage{etoolbox}
\usepackage{changepage}
\usepackage{tasks}
\usepackage{pgfplots}
\usepackage{fancyhdr}
\usepackage{stackengine}

\DeclareMathOperator{\arctg}{arctg}
\DeclareMathOperator{\tg}{tg}
\DeclareMathOperator{\sh}{sh}
\DeclareMathOperator{\ch}{ch}
\DeclareMathOperator{\sgn}{sgn}

\let\arctan\relax
\DeclareMathOperator{\arctan}{arctg}
\let\tan\relax
\DeclareMathOperator{\tan}{tg}

% Dodatkowe deklaracje:

% Koniec dodatkowych deklaracji

\pagestyle{fancy}

\lhead{Metody numeryczne - zestaw 0; 2020/2021}
% Tutaj proszę uzupełnić imię i nazwisko
\rhead{Wojciech Szlosek}

\begin{document}

\section*{Zadanie 3}

Na początek obliczmy pochodne funkcji $f(x) = \ln(x)$. Mamy $f^{'}(x) = x^{-1}, f^{''}(x) = -x^{-2}, f^{'''}(x) = 2x^{-3}$ i tak dalej, ogólnie:
$$f^{(k)}(x) = (-1)^{k-1}(k-1)!x^{-k} \ \ \ (k \geq 1)$$
Zauważmy też, że:
$$f^{(0)}(1) = f(1) = \ln(1) = 0$$
Zapiszmy zatem wzór Taylora:
$$\ln(x) = \sum_{k=1}^{n} \frac{(-1)^{k-1}(k-1)!c^{-k}}{k!} (x-c)^k + E_n(x)$$
Zgodnie z poleceniem, podstawiamy $c = 1$, ponadto uprośćmy nieco wyrażenie:
$$= \sum_{k=1}^{n} \frac{(-1)^{k-1}1^{-k}}{k!}(x-1)^k + E_n(x) = \sum_{k=1}^{n} \frac{(-1)^{k-1}}{k!}(x-1)^k + E_n(x)$$
gdzie:
$$E_n(x) = \frac{}{}$$

\section*{Zadanie 6b}
Założenie wstępne: w tym zadaniu za normę uznajmy zawsze normę nieskończoność.
$$||x||_{\infty} =  \stackunder{max}{i=1,...,n} |x_i| \implies ||A||_\infty = \underset{i = 1,...,n}{max} \sum_{j=1}^{n} |a_{i,j}|$$
Udowodnijmy prawą stronę implikacji, mając za założenie jej lewą stronę. \\
Z samej definicji normy dla dowolnej macierzy $A$ otrzymamy, że: 
$$||A||_{\infty} = \underset{||x||_{\infty} = 1}{max} ||Ax||_{\infty} = \underset{||x|| = 1}{sup} ||Ax||$$
Niech $m \in \{1, ..., n\}$. Zauważmy, że kolejne współrzędne wektora $Ax$ są postaci $a_{m_1}x_1 + a_{m_2}x_2 + a_{m_3}x_3 + ... + a_{m_n}x_n$. Stąd otrzymamy pierwszą poniższą równość, którą dodatkowo możemy "odwrócić" - zamienić ze sobą $sup$ i $max$:
$$\underset{||x|| = 1}{sup} ||Ax||=
    \underset{||x|| = 1}{sup} \underset{i = 1, ..., n}{max}\Big|\sum_{j=1}^{n}a_{ij}x_j \Big| =
     \underset{i = 1, ..., n}{max}\underset{||x|| = 1}{sup}\Big|\sum_{j=1}^{n}a_{ij}x_j \Big|
$$
W naszym $sup$ rozpatrujemy tylko wektory $x$, gdzie $|x_j| \leq 1$. Zastanówmy się teraz, kiedy osiągniemy największą możliwą sumę? Otóż wydarzy się to, gdy każda składowa $x$ będzie miała długość dokładnie 1 oraz gdy $x_j$ będzie tego samego znaku co $a_{ij}$. Otrzymamy zatem sumę wszystkich elementów macierzy $A$, a więc:
$$
    \underset{||x|| = 1}{sup} \underset{i = 1, ..., n}{max}\Big|\sum_{j=1}^{n}a_{ij}x_j \Big| =
     \underset{i = 1, ..., n}{max}\underset{||x|| = 1}{sup}\Big|\sum_{j=1}^{n}a_{ij}x_j \Big| = \underset{i = 1, ..., n}{max} \sum_{j=1}^{n}|a_{ij}| $$
Czego przecież mieliśmy dowieść.
\end{document}
