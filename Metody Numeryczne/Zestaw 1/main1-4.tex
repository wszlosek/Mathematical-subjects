\documentclass{article}
\usepackage[a4paper, margin={1in, 1in}]{geometry}
\usepackage[utf8]{inputenc}
\usepackage{polski}
\usepackage{mathtools}
\usepackage{amsfonts}
\usepackage{amssymb}
\usepackage{amsmath}
\usepackage{multicol}
\usepackage{paralist}
\usepackage{tabto}
\usepackage{graphicx}
\usepackage{etoolbox}
\usepackage{changepage}
\usepackage{tasks}
\usepackage{pgfplots}
\usepackage{fancyhdr}

\DeclareMathOperator{\arctg}{arctg}
\DeclareMathOperator{\tg}{tg}
\DeclareMathOperator{\sh}{sh}
\DeclareMathOperator{\ch}{ch}
\DeclareMathOperator{\sgn}{sgn}

\let\arctan\relax
\DeclareMathOperator{\arctan}{arctg}
\let\tan\relax
\DeclareMathOperator{\tan}{tg}

% Dodatkowe deklaracje:

% Koniec dodatkowych deklaracji

\pagestyle{fancy}

\lhead{Metody numeryczne, zestaw 1; 2020/2021}
% Tutaj proszę uzupełnić imię i nazwisko
\rhead{Wojciech Szlosek}

\begin{document}

\section*{Zadanie 4}
Ażeby pokazać, że obliczenie (wyznacznikowe) w czterocyfrowej arytmetyce dziesiętnej będzie znacznie różnić się od dokładnego wyniku, spróbujmy w niej obliczyć wartość $y$. \\
Mamy więc do obliczenia: 
$$y = \frac{31.69 \cdot 19.00 - 13.11 \cdot 45.00}{31.69 \cdot 5.89 - 13.11 \cdot 14.31}$$
Będziemy obracać się w arytmetyce czterocyfrowej, zatem będziemy musieć najpierw obliczyć po kolei poszczególne iloczyny wraz z wykonaniem zaokrąglenia. Czyli:
$$y = \frac{602.1 - 590.0}{186.7 - 187.6} = \frac{12.10}{-0.900} = -13.44$$
Pamiętajmy, że prawidłowy (dokładny) wynik to $y = -12.8$. Otrzymaliśmy niedokładny wynik, ponieważ prawie wszystkie operacje mnożenia/dodawania/dzielenia wiązały się u nas z zaokrągleniem - wszystkie zaokrąglenia wpłynęły na ostateczny wynik. \\ \\
Obliczmy wskaźnik uwarunkowania.
Niech $A = \left[ \begin{array}{cc}
31.69 & 14.31 \\
13.11 & 5.89 \\
\end{array} \right]$ \\ \\ \\
Z wykładu wiemy, że współczynnik uwarunkowania $c = ||A^{-1}|| \cdot ||A||$, \\ $det(A) = 31.69 \cdot 5.89 - 13.11 \cdot 14.31 = -0.95$. Mając wyznacznik, obliczamy macierz odwrotną (sprawa techniczna: tutaj już nie zaokrąglałem wyników w taki sposób jak wyżej - polecenie można interpretować różnie, wybrałem takowy sposób: bez dużych zaokrągleń), otrzymując:
$$A^{-1} = \left[ \begin{array}{cc}
-6.2 & 15.063 \\
13.8 & -33.358 \\
\end{array} \right]$$
Pora na obliczenie norm:
$$||A^{-1}||_1 = max(|-6.2|+|13.8|, |15.063|+|-33.358|) = max(20, 48.421) = 48.421$$
$$||A||_1 = max(|31.69|+|13.11|, |14.31|+|5.89|) = max(44.8, 20.2) = 44.8$$
Zatem obliczamy współczynnik uwarunkowania (ostateczny wynik w zaokrągleniu):
$$c = 48.421 \cdot 44.8 = 2169.26$$
Co jest dużą wartością. Nawet dosyć niewielkie błędy zaokrągleń będą powodować wielkie błędy wynikowe.
\end{document}