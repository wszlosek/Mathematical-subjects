\documentclass{article}
\usepackage[a4paper, margin={1in, 1in}]{geometry}
\usepackage[utf8]{inputenc}
\usepackage{polski}
\usepackage{mathtools}
\usepackage{amsfonts}
\usepackage{amssymb}
\usepackage{amsmath}
\usepackage{multicol}
\usepackage{paralist}
\usepackage{tabto}
\usepackage{graphicx}
\usepackage{etoolbox}
\usepackage{changepage}
\usepackage{tasks}
\usepackage{pgfplots}
\usepackage{fancyhdr}

\DeclareMathOperator{\arctg}{arctg}
\DeclareMathOperator{\tg}{tg}
\DeclareMathOperator{\sh}{sh}
\DeclareMathOperator{\ch}{ch}
\DeclareMathOperator{\sgn}{sgn}

\let\arctan\relax
\DeclareMathOperator{\arctan}{arctg}
\let\tan\relax
\DeclareMathOperator{\tan}{tg}

% Dodatkowe deklaracje:

% Koniec dodatkowych deklaracji

\pagestyle{fancy}

\lhead{Metody numeryczne, zestaw 1; 2020/2021}
% Tutaj proszę uzupełnić imię i nazwisko
\rhead{Wojciech Szlosek}

\begin{document}

\section*{Zadanie 2}

$$z = f(x,y) = x^2 + 8y^2$$ \\
a) $\epsilon_z = k_x \ \epsilon_x + k_y \ \epsilon_y$ \\ \\
Obliczamy/szacujemy $k_x$ i $k_y$:
$$k_x = \frac{\delta f(x,y)}{\delta x} \cdot \frac{x}{f(x,y)} = \frac{2x \cdot x}{x^2 +8y^2} \ \ \ \ \ |k_x| = |\frac{2x^2}{x^2 + 8y^2}| \leq |\frac{2x^2}{x^2}| = 2$$
$$k_y = \frac{16y \cdot y}{x^2 +8y^2} \ \ \ \ \ |k_y| = |\frac{16y^2}{x^2 + 8y^2}| \leq |\frac{16y^2}{8y^2}| = 2$$ 
Możemy zatem orzec:
$$\epsilon_z \leq 2 \cdot |\epsilon_x| + 2 \cdot |\epsilon_y|$$ \\
b) Analogicznie do podpunktu a) obliczamy $k_{zx}$ oraz $k_{zy}$, otrzymując (ponownie):
$$|k_{zx}| \leq |\frac{2x^2}{x^2}| = 2 \leq 10 $$
$$|k_{zy}| \leq |\frac{16y^2}{8y^2}| = 2 \leq 10 $$
Stąd możemy stwierdzić, że są one dobrze uwarunkowane (wartość wskaźników - zgodnie z poleceniem - nie jest większa od 10). \\ \\
c) Obliczmy błąd nieunikniony, wykorzystując pochodne po $x$ i po $y$:
$$\Delta^{0}_{z} = \Big(
\left[ \begin{array}{cc}
|2x| & |16y|\\
\end{array} \right] \cdot
\left[ \begin{array}{c}
|x| \\
|y| \\
\end{array} \right] + |x^2+8y^2| \Big) \epsilon_{masz} = (2x^2 + 16y^2 +x^2+8y^2) \epsilon_{masz} = $$
$$= (3x^2 + 24y^2) \epsilon_{masz}$$ \\
d) $\phi(x,y) = \phi_2 \ \circ \ \phi_1 \ \circ \ \phi_1, \ \ \phi_1(a,b) = (b,a^2), \ \phi_2(c, d) = c+8d$.
Mamy:
$$fl(\phi_2 \ \circ \ \phi_1 \ \circ \ \phi_1) = fl(\phi_2) \ \circ \ fl(\phi_1) \ \circ \ fl(\phi_1) (x,y) = fl(\phi_2) \ \circ \ fl(\phi_1) (y, x^2(1+\epsilon_1)) = $$
$$= fl(\phi_2)(x^2(1+\epsilon_1), y^2(1+\epsilon_2)) = (x^2(1+\epsilon_1)+8y^2(1+\epsilon_2))(1+\epsilon_3) = $$
$$= (x^2+8y^2) + (x^2 \cdot \epsilon_1) + (8y^2 \cdot \epsilon_2) + (\epsilon_3(x^2+8y^2)) + ...$$ \\
Zauważmy teraz, że pierwszy nawias równy jest $z$. Z kolei moduł każdego kolejnego nawiasu jest niewiększy od modułu z $\Delta^{0}$, które to obliczyliśmy w poprzednim podpunkcie tego zadania. \\
Zatem możemy orzec, że ten algorytm jest stabilny numerycznie.
\end{document}