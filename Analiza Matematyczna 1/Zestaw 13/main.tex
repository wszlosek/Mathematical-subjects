\documentclass{article}
\usepackage[utf8]{inputenc}
\usepackage{polski}
\usepackage{pgfplots}

\title{AM1 - Zestaw 13}
\author{Wojciech Szlosek}
\date{May 2020}

\begin{document}

\maketitle

\section{Zadanie 1, (d)}

Z definicji wartość średnia funkcji $f$ na przedziale $[a,b]$ wyraża się wzorem:
$$f_{sr} = \frac{1}{b-a} \int^{b}_{a}f(x)dx$$

Zatem w naszym przykładzie:
$$P_{sr} = \frac{1}{\frac{1}{2}}\int^{\frac{1}{2}}_{0}x\sqrt{1-x^2} dx = [\frac{x^{2}\sqrt{1-x^2}-\sqrt{1-x^2}}{3}]^{\frac{1}{2}}_{0} =\frac{-1}{4} \sqrt{\frac{3}{4}} + \frac{1}{3} $$

\section{Zadanie 2, (a)}

Wystarczy uzasadnić, że funkcja podcałkowa $f(x) = e^{x^{2}} \sin(-x)$ jest nieparzysta. Dla rzeczywistego $x$ i $-x$ mamy:
$$f(-x) = e^{x^{2}} \sin(-x) = -e^{x^2}\sin(x) = -f(x)$$

Zatem rzeczywiście, całka z treści zadania jest równa 0.

\section{Zadanie 3, (a)}

Mamy:
$$F(x) = \int^{x}_{-1}f(t) dt = \left\{ \begin{array}{rl}
\int^{x}_{-1} (1) dt & \textrm{dla $-1 \leq x \leq 0$} \\ \int^{x}_{-1} (\frac{3x}{2}) dt & \textrm{dla $0 < x \leq 2$} \end{array} = \left\{ \begin{array}{rl}
x+1 & \textrm{dla $-1 \leq x \leq 0$} \\ \frac{3x^2}{4}+1 & \textrm{dla $0 < x \leq 2$} \end{array}$$

Poniżej znajdują się wykresy funkcji $f$ i $F$.

\begin{tikzpicture}
    \begin{axis}[ 
axis lines=middle, 
enlargelimits=true,
ymin=-2, 
ymax=4, 
xlabel=$x$,
ylabel=$f(x)$, 
xtick={-1,0,1,2}, 
ytick={-2,-1,0,1,2,3}
]
\addplot+[domain=-1:0,mark=none,samples=200,color=blue] {1};
\addplot+[domain=0:2,mark=none,samples=200,color=blue] {3*x/2};

     \end{axis}
\end{tikzpicture}

\begin{tikzpicture}
    \begin{axis}[ 
axis lines=middle, 
enlargelimits=true,
ymin=-2, 
ymax=4, 
xlabel=$x$,
ylabel=$F(x)$, 
xtick={-1,0,1,2}, 
ytick={-2,-1,0,1,2,3}
]
\addplot+[domain=-1:0,mark=none,samples=200,color=blue] {x+1};
\addplot+[domain=0:2,mark=none,samples=200,color=blue] {(3*x^2)/4 + 1};

     \end{axis}
\end{tikzpicture}


\end{document}