\documentclass{article}
\usepackage[utf8]{inputenc}

\title{AM - Zestaw 5}
\author{Wojciech Szlosek}
\date{March 2020}

\begin{document}

\maketitle

\section{Zadanie 1}

\subsection{a)}

$f(x) = |x-1|, x_{0} = 1$
$$f^{'}(1) = \lim_{x \to 1} \frac{f(x)-f(1)}{x-1} = \lim_{x \to 1} \frac{|x-1|-0}{x-1}$$
$$\lim_{x \to 1^{+}}\frac{|x-1|}{x-1}=1$$
$$\lim_{x \to 1^{-}}\frac{|x-1|}{x-1}=-1$$
$$1 \neq -1$$
(Odp.) Nie istnieje pochodna tej funkcji w $x_{0}$.

\section{Zadanie 3}

\subsection{c)}

$h(x) = |\sin{x}|, x_{0} = \pi$
$$h^{'}_{-}(\pi) = \lim_{x \to \pi^{-}}\frac{h(x)-h(\pi)}{x-\pi} = \lim_{x \to \pi^{-}}\frac{|\sin{x}|-|\sin{\pi}|}{x-\pi} = \lim_{x \to \pi^{-}}\frac{\sin{x}}{x-\pi} = \lim_{x \to \pi^{-}}\cos{x} = -1$$
$$h^{'}_{+}(\pi) = \lim_{x \to \pi^{+}}\frac{h(x)-h(\pi)}{x-\pi} = \lim_{x \to \pi^{+}}\frac{|\sin{x}|-0}{x-\pi} = \lim_{x \to \pi^{+}}\frac{-\sin{x}}{x-\pi} = -\lim_{x \to \pi^{+}}\cos{x} = -(-1) = 1$$
$$-1 \neq 1$$
(Odp.) Nie istnieje $h^{`}(\pi)$.

\subsection{Zadanie 7}

\subsection{b)}
$$(f^{-1})(e+1); f(x) = x + \ln{x}$$
Funkcja jest ciągła i rosnąca w zbiorze liczb rzeczywistych. Ponadto $f(e) = 1 + e$. Zatem argument $x = e$ tejże funkcji jest jedynym rozwiązaniem równania $x + \ln{x} = 1 + e$. Funkcja spełnia założenia twierdzenia o pochodnej funkcji odwrotnej, zatem: \newline
$$(f^{-1})^{'}(e+1) = \frac{1}{f^{'}(e)} = \frac{1}{(x+\ln{x})^{'}}_{x=e} = \frac{1}{1+\frac{1}{e}} = \frac{e}{e+1}$$

\end{document}