\documentclass{article}
\usepackage[utf8]{inputenc}

\title{AM1 - Zestaw 3}
\author{Wojciech Szlosek}
\date{March 2020}

\begin{document}

\maketitle

\section{Zadanie 1, b)} 

$$\lim_{x \to \infty}({3}^{-x}+1) = 1 $$ 
\newline
Mamy pokazać, że: \newline

$ \bigwedge_{x_{n}}$ $[( \lim_{n \to \infty} x_{n} = \infty ) \Rightarrow$ $(\lim_{n \to \infty} (3^{-x_{n}}+ 1) = 1)]$ \newline \newline
Niech $x_{n}$ będzie dowolnym ciągiem spełniającym warunek: $\lim_{x \to \infty} = \infty $. Wówczas: 

$$ \lim_{n \to \infty} (3^{-x_{n}}+ 1) = \lim_{n \to \infty} ({(\frac{1}{3})}^{x_{n}} + 1) = \lim_{n \to \infty} ({(\frac{1}{3}}^{\infty} + 1) = 0 + 1 = 1 $$

\section{Zadanie 6, c)}
 
 $$\lim_{x \to \infty} 2^{x}(2+\cos x) = \infty$$ \newline

        Zauważmy, że skoro $\cos x \in [-1, 1]$, to: \newline \newline
        $$2^x(2+\cos x) \ge 2^x$$ \newline
        
        Widzimy, że prawa strona (ograniczenie dolne) tejże nierówności dąży do $\infty$.
        Zatem z twierdzenia o dwóch funkcjach, $2^{x}(2+\cos x) $ również dąży do $\infty$. Więc rzeczywiście $$\lim_{x \to \infty} 2^{x}(2+\cos x) = \infty$$. \newline
        Co nakazano dowieść.
        
\section{Zadanie 7, k)}

$$\lim_{x \to 0} \frac{\ln(1+\sqrt[3]{x})}{x} = \lim_{x \to 0} (\frac{\ln(1+\sqrt[3]{x})}{\sqrt[3]{x}} \cdot \frac{\sqrt[3]{x}}{x}) = \lim_{x \to 0} (\frac{\ln(1+\sqrt[3]{x})}{\sqrt[3]{x}} \cdot \frac{1}{\sqrt[3]{x^{2}}}) = 1 \cdot \infty = \infty $$ \newline

W trakcie rozwiązania wykorzystałem własność: \newline

$\lim_{u \to 0} {\frac{\ln(1+u)}{u}} = 1$

\end{document}
