\documentclass{article}
\usepackage[utf8]{inputenc}
\usepackage{polski}
\usepackage{pgfplots}

\title{AM 1 - Zestaw 14}
\author{Wojciech Szlosek}
\date{May 2020}

\begin{document}

\maketitle

\section{Zadanie 1, (e)}

$$x = y^3 -y; x = 0$$
Szukamy punktów wspólnych krzywych, $y^3 - 3 = 0$, stąd: $y_1 = -1$, $y_2 = 0$, $y_3 = 1$.
Podzielmy pole $D$ na dwie części (gdy $0 \leq y \leq 1$ oraz $-1 \leq y \leq 0$).
$$|D_1| = \int_{a}^{b} (g(y)-f(y))dy = \int_{0}^{1} (-y^3 +y) dy = [\frac{-y^4 +2y^2}{4}]_{0}^{1} = \frac{1}{4}$$
Analogicznie:
$$|D_2| = \int_{0}^{-1} (-y^3 +y) dy = \frac{1}{4}$$
Szukane $|D| = |D_1| + |D_2| = \frac{1}{2}$

\section{Zadanie 2, (b)}

$f(x) = ch x$; $f^{'}(x) = sh(x) $ $a = 0 \leq x \leq 1 = b$; $s$ - szukana długość krzywej \newline
f ciągła na $[0,1]$.
Ponadto po drodze skorzystam z faktu, że $ch^{2} x - sh^{2} x = 1$ oraz $ ch x \geq 0$.

$$|s| = \int_{a}^{b} \sqrt{1+(f^{'}(x))^{2}} dx = \int_{0}^{1} \sqrt{1+sh^2(x)} dx = \int_{0}^{1} ch x dx = sh x |_{0}^{1}$$
$$ = sh 1 - sh 0 = \frac{e - e^{-1}}{2} (>0)$$

\section{Zadanie 3, (b)}

$$T: a = 1 \leq x \leq 3; 0 \leq y = f(x) \leq \frac{1}{x}; OY$$
f nieujemna i ciągła na $[1,3]$.

$$|V| = 2\pi \int_{a}^{b} xf(x)dx = 2\pi \int_{1}^{3} x \cdot \frac{1}{x} dx = 2\pi \int_{1}^{3} 1 dx = 2\pi \cdot 2 = 4\pi$$

\end{document}