\documentclass{article}
\usepackage[utf8]{inputenc}
\usepackage{polski}

\title{AM1 - Zestaw 12}
\author{Wojciech Szlosek}
\date{May 2020}

\usepackage{natbib}
\usepackage{graphicx}

\begin{document}

\maketitle

\section{Zadanie 2, (e)}

$$\int_{0}^{\frac{\pi}{2}} (e^{2x} \cosx) dx = [\frac{e^{2x}(\sin x+2\cos x)}{5}]^{\frac{\pi}{2}}_{0} = \frac{e^{\pi}(1+0)}{5} - \frac{1(0+2)}{5} = \frac{e^{\pi}-2}{5}$$

\section{Zadanie 3, (a)}

W tym zadaniu wykorzystamy wiadomy wzór: $$\int_{a}^{b} f(x) dx = \lim_{n \to \infty}[\frac{b-a}{n} \sum_{i=1}^{n} f(a+i\frac{b-a}{n})] $$

Zatem do tego wzoru możemy przyjąć $[a,b]=[0,1]$ oraz $f(x) = x^3$, gdzie $f$ jest funkcją ciągłą i całkowalną na podanym przedziale. Wówczas mamy:

$$ \lim_{n \to \infty} \frac{1^3 + 2^3 +...+ n^3}{n^4} = \lim_{n \to \infty} [\frac{1}{n} \sum_{i=1}^{n} (\frac{i}{n})^3] = \int_{0}^{1} x^3 dx = [\frac{x^4}{4}]_{0}^{1} = \frac{1}{4}$$
Co nakazano dowieść.

\section{Zadanie 7, (a)}

$$\int_{0}^{1} e^{x} \sqrt[6]{1+x^3} dx$$

Zauważmy, że dla każdego $x$ z przedziału $[0,1]$ prawdziwa jest zależność:

$$e^0 \sqrt[6]{1+0} \leq e^x \sqrt[6]{1+x^3} \leq e^1 \sqrt[6]{1+1}$$

Zatem:

$$1 \leq \int_{0}^{1} e^x \sqrt[6]{1+x^3} \leq e \sqrt[6]{2}$$

\end{document}