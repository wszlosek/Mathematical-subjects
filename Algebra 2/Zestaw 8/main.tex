\documentclass{article}
\usepackage[utf8]{inputenc}

\title{Algebra 2 - Zestaw 8}
\author{Wojciech Szlosek}
\date{April 2020}

\begin{document}

\maketitle

\section{Zadanie 6, (a)}

$$(x,y) = 2x_{1}y_{1}-x_{1}y_{2}-x_{2}y_{1}+x_{2}y_{2}$$
Udowodnimy poprzez sprawdzenie pięciu wymaganych warunków.

$$1. (x,y) = 2x_{1}y_{1}-x_{1}y_{2}-x_{2}y_{1}+x_{2}y_{2} = 2y_{1}x_{1}-y_{2}x_{1}-y_{1}x_{2}+x_{2}y_{2}=(y,x)$$

$2. (x+y,z) = 2(x_{1}+y_{1})z_{1}-(x_{1}+y_{1})z_{2}-(x_{2}+y_{2})z_{1}+(x_{2}+y_{2})z_{2} = (2x_1 z_1 -x_1 z_2 - x_2 z_1 + x_2 z_2) + (2y_1 z_1 -y_1 z_2 -y_2 z_1 + y_2 z_2) = (x,z) + (y,z)$ 

$$3. (ax,y) = 2(ax_1 )y_1 - (ax_1 )y_2 - (ax_2 )y_1 + (ax_2 )y_2 = a(2x_{1}y_{1}-x_{1}y_{2}-x_{2}y_{1}+x_{2}y_{2}) = a(x,y)$$

$4.(x,x)=2x_1^2 -2x_1 x_2 + x_2^2 \geq 0$, delta = $-8x_2^2 \leq 0$, zatem zaiste $(x,x) \geq 0$

$$5.(x,x) = 0 \Rightarrow x = 0$$
delta = $-4x_2^2 = 0 \Rightarrow x_2 = 0$, $x_1 = 0$ \newline

Dowiodłem tych pięciu warunków, zatem udowodniłem całości, cnd.

\section{Zadanie 7, (a)}

$$(x,y) = 2x_1 y_1 + 3x_1 y_2 - x_2 y_1 + 5x_2 y_2$$

Zauważmy, że $(x,y) \neq (y,x)$, np. dla $x=(1,2)$ oraz $y = (1,0)$. Tym kontrprzykładem udowodniłem, że nie jest ta funkcja iloczynem skalarnym w rozważanych przestrzeniach wektorowych.
\end{document}