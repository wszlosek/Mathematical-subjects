\documentclass{article}
\usepackage[utf8]{inputenc}
\usepackage{polski}

\title{Algebra 2 - Zestaw 12}
\author{Wojciech Szlosek}
\date{May 2020}


\begin{document}

\maketitle

\section{Zadanie 1, (a)}

$ \forall x \in R $  $0 \cdot x = 0$ i $x \cdot 0 = 0$ \newline

Dowód: \newline

Ponieważ $0 = 0+0$, więc (z def. pierścienia): $x \cdot (0+0) = x\cdot0 + x \cdot 0$. Skąd (z prawa skracania w grupach) mamy, że $x \cdot 0 = 0$. Analogicznie, $0 \cdot x = (0+0) \cdot x = 0\cdot x + 0\cdot x$ Zatem $0 \cdot x = 0$. Udowodniłem więc wejściowe własności z treści zadania. CND.

\section{Zadanie 1, (b), (c)}

(b) $ \forall x,y \in R$  $(-x)\cdot y = -(x \cdot y)$ i $x \cdot(-y) = -(x \cdot y)$ \newline

Dowód: (wykorzystamy własność z podpunktu (a) \newline

$x \cdot y + (-x) \cdot y = ((-x)+x)\cdot y = 0 \cdot y = 0$ [(a)]. Mamy zatem: $(-x) \cdot y = - (x \cdot y)$. \newline

Analogicznie dowód wygląda dla $x \cdot (-y) = - (x\cdot y)$. CND. \newline \newline

(c) $ \forall x,y \in R$ $(-x) \cdot (-y) = (x\cdot y)$ \newline

Dowód: \newline

Użyjemy wniosku z poprzedniego podpunktu. Mamy: $(-x) \cdot (-y) = - (x \cdot (-y)) = -(-(x\cdot y)) = (x \cdot y)$ CND. \newline \newline \newline 

\section{Zadanie 5, (a)}
$$Z_5$$
Możemy sobie wytłumaczyć, że w tym podpunkcie elementami odwracalnymi tutaj będą te liczby dla których największy wspólny dzielnik z 5 wynosi 1. I tak u nas: \newline

Elementy odwracalne: $ \{ 1,2,3,4 \}$. \newline

Brak dzielników zera. \newline

Dany pierścień jest ciałem.

\section{Zadanie 5, (b)}
$$Z_8$$
Analogicznie do podpunktu (a). Zatem elementami odwracalnymi są: $\{ 1,3,5,7 \}$. Dzielnikami zera są $\{2,4,6\}$. \newline
Dany pierścień nie jest ciałem.

\section{Zadanie 7}

$$x \cdot x^2 - (x^3 -5) = 5 \in I$$
Niech $J \subseteq R$ będzie ideałem generowanym przez zbiór $\{ 5,x \}$. $5 \in I$, $x \in I$ - zatem $J \subseteq I$. \newline \newline
Przypuśćmy, że $J = (z)$. Wtedy $z |_{Z(x)} 5$. Zatem $z = 1$ lub $z = 5$. Warto zauważyć, że $J = \{ a_{n}x^{n} + a_{n-1}x^{n-1}...+a_{1}x+5a_{0}; a_{n},a_{n-1},...,a_{1},a_{0} \in Z \}$ (!) \newline \newline
Jeżeli $z = 1$, to $J = Z(x)$, co stoi w sprzeczności z (!), ponieważ $1 \notin J$ Czyli sprzeczność. \newline
Jeżeli $z = 5$, to $x \notin J = 5Z(x)$. Sprzeczność. \newline \newline

(Odp.) Zatem nie jest ideałem głównym.

\end{document}