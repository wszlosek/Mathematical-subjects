\documentclass{article}
\usepackage[utf8]{inputenc}
\usepackage{polski}

\title{Algebra 2 - Zestaw 7}
\author{Wojciech Szlosek}
\date{March 2020}

\begin{document}

\maketitle

\section{Zadanie 1}

\subsection{(a)}

Wartość własna $\lambda$ rzeczywistej macierzy A stopnia n jest pierwiastkiem wielomianu charakterystycznego tej macierzy i wyznaczamy ją z warunku $det(A-I\lambda) = 0$. Wektor własny $v = (x_1,x_2,...)$ odpowiadający wartości własnej $\lambda$ jest niezerowym rozwiązaniem jednorodnego układu równań:

$$(A-I\lambda)\left[\begin{array}{c}x_1\\.\\.\\x_n\end{array}\right] = \left[\begin{array}{c}0\\.\\.\\0\end{array}\right]$$
$$-------------------------------$$

$$A = \left[\begin{array}{cc}2&-1\\1 &4\end{array}\right]$$
$$det(A-I\lambda) = det \left[\begin{array}{cc}2-\lambda & -1\\1 & 4-\lambda\end{array}\right] = \lambda^2 -6\lambda + 9 = 0$$
$$\lambda = 3$$ \newline

Wyznaczmy wektor własny $v = (x,y)$.

$$(A-I\lambda)\left[\begin{array}{c}x\\y\end{array}\right] = \left[\begin{array}{cc}-1&-1\\1&1\end{array}\right]\left[\begin{array}{c}x\\y\end{array}\right]=\left[\begin{array}{c}0\\0\end{array}\right]$$
\newline

Zatem: $x+y=0 \Rightarrow y=-x$
$$v = (x,-x)$$
$$\underline{W_3 = lin\{(1,-1)\}}$$


\subsection{(c)}

Wartość własna $\lambda$ rzeczywistej macierzy A stopnia n jest pierwiastkiem wielomianu charakterystycznego tej macierzy i wyznaczamy ją z warunku $det(A-I\lambda) = 0$. Wektor własny $v = (x_1,x_2,...)$ odpowiadający wartości własnej $\lambda$ jest niezerowym rozwiązaniem jednorodnego układu równań:

$$(A-I\lambda)\left[\begin{array}{c}x_1\\.\\.\\x_n\end{array}\right] = \left[\begin{array}{c}0\\.\\.\\0\end{array}\right]$$
$$-------------------------------$$

$$A = \left[\begin{array}{cc}\sqrt{3}&-1\\1&\sqrt{3}\end{array}\right]$$
$$det(A-I\lambda) = det \left[\begin{array}{cc}\sqrt{3}-\lambda & -1\\1 & \sqrt{3}-\lambda\end{array}\right] = \lambda^2 -2\sqrt{3}\lambda + 4 = 0$$ \newline

To równanie nie ma pierwiastków rzeczywistych $(\Delta<0)$, co oznacza, że macierz A nie ma rzeczywistych wartości własnych.


\subsection{(f)}

Wartość własna $\lambda$ rzeczywistej macierzy A stopnia n jest pierwiastkiem wielomianu charakterystycznego tej macierzy i wyznaczamy ją z warunku $det(A-I\lambda) = 0$. Wektor własny $v = (x_1,x_2,...)$ odpowiadający wartości własnej $\lambda$ jest niezerowym rozwiązaniem jednorodnego układu równań:

$$(A-I\lambda)\left[\begin{array}{c}x_1\\.\\.\\x_n\end{array}\right] = \left[\begin{array}{c}0\\.\\.\\0\end{array}\right]$$
$$-------------------------------$$

$$A = \left[\begin{array}{ccc}0&1&0\\-4&4&0\\-2&1&2\end{array}\right]$$

$$det(A-I\lambda) = det \left[\begin{array}{ccc}-\lambda&1&0\\-4&4-\lambda&0\\ -2&1&2-\lambda\end{array}\right] = -\lambda^3 + 6\lambda^2 -12\lambda + 8 = 0 \Rightarrow (x-2)^3 = 0$$

$$\lambda = 2$$
Wyznaczmy wektor własny $v = (x,y,z)$: \newline

$$(A-I\lambda) \left[\begin{array}{ccc}x\\y\\z\end{array}\right] = \left[\begin{array}{ccc}-2&1&0\\-4&2 &0\\-2&1&0\end{array}\right]  \left[\begin{array}{c}x\\y\\z \end{array}\right] = \left[\begin{array}{c}0\\0\\0 \end{array}\right]$$

Stąd: $-2x+y = 0 \Rightarrow y = 2x$
\newline

$$v = (x,2x,z)$$
$$\underline{W_2 = lin\{(1,2,0),(0,0,1)\}}$$


\section{Zadanie 2}

\subsection{(e)}

$$A = \left[\begin{array}{ccc}i&i&i\\1&1&1\\2&2&2\end{array}\right]$$

$$det(A-I\lambda) = det\left[\begin{array}{ccc}1-\lambda&i&i\\1&1-\lambda&1\\2&2&2-\lambda\end{array}\right] = \lambda^2(-\lambda+(3+i)) = 0$$

$$\lambda_1 = 0 ; \lambda_2 = 3+i$$

$$(A-I\lambda_1)\left[\begin{array}{c}x\\y\\z\end{array}\right]=\left[\begin{array}{ccc}i&i&i\\1&1 &1\\2&2&2\end{array}\right]\left[\begin{array}{c}x\\y\\z \end{array}\right]= \left[\begin{array}{c}0\\0 \\0\end{array}\right] \Rightarrow x+y+z = 0 \Rightarrow z = -x-y $$
$$\underline{v_1 = (x,y,-x-y)}$$ \newline

$$(A-I\lambda_2)\left[\begin{array}{c}x\\y\\z\end{array}\right]=\left[\begin{array}{ccc}-3&i&i\\1&-2-i &1\\2&2&-1-i\end{array}\right]\left[\begin{array}{c}x\\y\\z \end{array}\right]= \left[\begin{array}{c}0\\0 \\0\end{array}\right] \Rightarrow x = iy, z = 2y $$
$$\underline{v_2 = (iy,y,2y)}$$
\newline
\centering
(Odp.)
$$W_0 = lin_c\{(1,0,-1),(0,1,-1)\}$$
$$W_{3+i} = lin_c \{(i,1,2)\}$$

\end{document}