\documentclass{article}
\usepackage[utf8]{inputenc}
\usepackage{polski}

\title{Algebra 2 - Zestaw 13}
\author{Wojciech Szlosek}
\date{May 2020}

\begin{document}

\maketitle

\section{Zadanie 3}

$\left\{ \begin{array}{ll}
x \equiv 4$ (mod 3)$\\
x \equiv 10$ (mod 4) $\\
x \equiv 8$ (mod 7) $
\end{array} \right $ \newline \newline \newline
Zmieńmy nieco treść zadania (na wersję równoznaczną), ale lepiej działającą na wyobraźnię. \newline
Na osi liczbowej zaznaczone są liczby całkowite. W punktach $x_A = 4$, $x_B = 10$, $x_C = 8$ znajdują się kolejno pionki A, B, C. Każdy z nich może poruszać się w lewo lub prawo, wykonując kroki o długościach kolejno $n_A = 3$, $n_B = 4$, $n_C = 7$. Wyznacz wszystkie punkty, w których mogą się spotkań wszystkie trzy pionki jednocześnie. \newline \newline
W zbiorze liczb całkowitych zdefiniujmy trzy ideały: $I_1 = 3Z$, $I_2 = 4Z$, $I_3 = 7Z$. \newline
3, 4, 7 są liczbami względnie pierwszymi (NWD(3,4) = NWD(4,7) = NWD(7,3) = 1.
Stwierdzamy zatem, że:
$$I_1 + I_2 = Z; I_2 + I_3 = Z; I_1 + I_3 = Z$$
Dochodzimy teraz do następujących wniosków:

$$x - x_A \in 3Z \Leftrightarrow E k_A \in Z: x = x_A + 3k_A$$
$$x - x_B \in 4Z \Leftrightarrow E k_B \in Z: x = x_B + 4k_B$$
$$x - x_C \in 7Z \Leftrightarrow E k_C \in Z: x = x_C + 7k_C$$
Liczba $x$ jest więc numerem pola, do którego pionek $A$ dotrze z pola $x_A$ po wykonaniu $k_A$ kroków o długości 3, analogicznie dla $B$ oraz $C$. Tym sposobem wszystkie pionki spotkają się na polu o numerze $x$, wyznaczmy zatem tenże.
$$I_1 \cap I_2 = 3 \cdot 4Z = 12Z$$
$$I_2 \cap I_3 = 4 \cdot 7Z = 28Z$$
$$I_1 \cap I_3 = 3 \cdot 7Z = 21Z$$

$$I_1 + I_2 \cap I_3 = 3Z + 28Z = Z$$
$$I_2 + I_1 \cap I_3 = 4Z + 21Z = Z$$
$$I_3 + I_1 \cap I_2 = 7Z + 12Z = Z$$

Przedstawmy liczby $x_A, x_B, x_C$ w następujących postaciach:
$$x_A = a_1 + b_1; a_1 \in 3Z, b_1 \in 28Z$$
$$x_B = a_2 + b_2; a_2 \in 4Z, b_2 \in 21Z$$
$$x_C = a_3 + b_3; a_3 \in 7Z, b_3 \in 12Z$$ \newline
Aby to osiągnąć, zacznijmy od tego, by liczbę 1 przedstawić w takiej postaci. Szukamy więc takich liczb $m_1, k_1, m_2, k_2, m_3, k_3 \in Z$, takich że:

$$\left\{ \begin{array}{ll}
1 = 3m_1 + 28k_1\\
1 = 4m_2 + 21k_2 \\
1 = 7m_3 + 12k_3 
\end{array} \right $$
Znalezione rozwiązania to:
$$\left\{ \begin{array}{ll}
1 = (-93) \cdot 3 + 10 \cdot 28\\
1 = (-5) \cdot 4 + 1 \cdot 21 \\
1 = (-5) \cdot 7 + 3 \cdot 12
\end{array} \right $$
Teraz możemy przedstawić $x_A,x_B,x_C$ w postaci:
$$\left\{ \begin{array}{ll}
x_A = 4 = 4 \cdot 1 = 4 \cdot ((-93) \cdot 3 + 10 \cdot 28) = (-372) \cdot 3 + 40 \cdot 28\\
x_B = 10 \cdot 1 = 10 \cdot ((-5) \cdot 4 + 1 \cdot 21) = (-50) \cdot 4 + 10 \cdot 21 \\
x_C = 8 \cdot 1 = 8 \cdot ((-5) \cdot 7 + 3 \cdot 12) = (-40) \cdot 7 + 24 \cdot 12
\end{array} \right $$ \newline
$$x = b_1 + b_2 + b_3 = 40 \cdot 28 + 10 \cdot 21 + 24 \cdot 12 = 1618$$

Liczba ta powinna być rozwiązaniem naszego zadania, dla sprawdzenia obliczmy:
$$1618 - 1614 = 3 \cdot 538 \Leftrightarrow x = x_A + 3 \cdot 538$$, analogicznie:
$$x = x_B + 4 \cdot 402$$
$$x = x_C + 7 \cdot 230$$
Oznacza to, że pionek $A$ dojdzie na pole $x = 1618$ po wykonaniu 538 kroków w prawo (itd. dla $B$ oraz $C$). Ponadto, jeżeli punkt $y \in Z$ jest innym punktem spotkań wszystkich trzech pionków, to musi spełniać warunek:
$$x - y \in I_1 \equiv I_2 \equiv I_3 = 3 \cdot 4 \cdot 7Z = 84Z$$
$$y = 1618 + 84k, k \in Z$$
\newline
(Odp.) Pole na których mogą się spotkać wszystkie trzy pionki równocześnie mają numery postaci $y = 1618 + 84k$, $k \in Z$.

\section{Zadanie 4, (a)}

Będziemy korzystać z tabelki działań w $Z_{4}[X]$ \newline
$$ \newline
\begin{tabular}{lllll}
+ & 0 & 1 & 2 & 3 \\
0 & 0 & 1 & 2 & 3 \\
1 & 1 & 2 & 3 & 0 \\
2 & 2 & 3 & 0 & 1 \\
3 & 3 & 0 & 1 & 2
\end{tabular}  $$   
\newline $$ \newline
\begin{tabular}{lllll}
\cdot & 0 & 1 & 2 & 3 \\
0 & 0 & 0 & 0 & 0 \\
1 & 0 & 1 & 2 & 3 \\
2 & 0 & 2 & 0 & 2 \\
3 & 0 & 3 & 2 & 1
\end{tabular} $$
\newline 

$$f(x) = 2x^{3}+3x^{2}+3x+3$$
$$g(x) = 3x^{2}+x+1$$
\newline
$\begin{array}{lll}
(2x^3 - 3x^2 + 3x + 3)  :  (3x^{2}+x+1)  = 2x+3 \\
\underline{-(2x^3 + 2x^2 +2x)} & &  \\
\qquad x^2 + x +3 & & \\
 \ \ \underline{-(x^2 +3x+3)} & &\\
\qquad  \quad r = 2x & &
\end{array} $
$$deg(r) = deg(q) = 1$$
$$f(x)= qg + r = (2x+3)(3x^2 +x+1) + 2x$$
Nie istnieją jednak wielomiany $q,r \in Z_4 [X]$, takie że $deg(r)<deg(q)$ i $f = qg + r$.

\newpage

\section{Zadanie 4, (c)}

Będziemy korzystać z tabelki działań w $Z_{4}[X]$ \newline
$$ \newline
\begin{tabular}{lllll}
+ & 0 & 1 & 2 & 3 \\
0 & 0 & 1 & 2 & 3 \\
1 & 1 & 2 & 3 & 0 \\
2 & 2 & 3 & 0 & 1 \\
3 & 3 & 0 & 1 & 2
\end{tabular}  $$   
\newline $$ \newline
\begin{tabular}{lllll}
\cdot & 0 & 1 & 2 & 3 \\
0 & 0 & 0 & 0 & 0 \\
1 & 0 & 1 & 2 & 3 \\
2 & 0 & 2 & 0 & 2 \\
3 & 0 & 3 & 2 & 1
\end{tabular} $$
\newline 

$$f(x) = 2x^{3}+3x^{2}+3x+3$$
$$g(x) = 2x^{2}+x+1$$
\newline
$\begin{array}{lll}
(2x^3 + 3x^2 + 3x + 3)  :  (2x^{2}+x+1)  = x+1 \\
\underline{-(2x^3 + x^2 +x)} & &  \\
\qquad 2x^2 + 2x +3 & & \\
 \ \ \underline{-(2x^2 +x+1)} & &\\
\qquad  \quad r = x+2 & &
\end{array} $
$$deg(r) = deg(q) = 1$$
$$f(x)= qg + r = (x+1)(2x^2 +x+1) + x+2$$
Nie istnieją jednak wielomiany $q,r \in Z_4 [X]$, takie że $deg(r)<deg(q)$ i $f = qg + r$.

\newpage

\section{Zadanie 5}

Będziemy korzystać z tabelki działań w $Z_7 [X]$.

$$
\centering
\begin{tabular}{llllllll}
+ & 0 & 1 & 2 & 3 & 4 & 5 & 6 \\
0 & 0 & 1 & 2 & 3 & 4 & 5 & 6 \\
1 & 1 & 2 & 3 & 4 & 5 & 6 & 0 \\
2 & 2 & 3 & 4 & 5 & 6 & 0 & 1 \\
3 & 3 & 4 & 5 & 6 & 0 & 1 & 2 \\
4 & 4 & 5 & 6 & 0 & 1 & 2 & 3 \\
5 & 5 & 6 & 0 & 1 & 2 & 3 & 4 \\
6 & 6 & 0 & 1 & 2 & 3 & 4 & 5
\end{tabular} $$ \newline
$$
\begin{tabular}{llllllll}
\cdot & 0 & 1 & 2 & 3 & 4 & 5 & 6 \\
0 & 0 & 0 & 0 & 0 & 0 & 0 & 0 \\
1 & 0 & 1 & 2 & 3 & 4 & 5 & 6 \\
2 & 0 & 2 & 4 & 6 & 1 & 3 & 5 \\
3 & 0 & 3 & 6 & 2 & 5 & 1 & 4 \\
4 & 0 & 4 & 1 & 5 & 2 & 6 & 3 \\
5 & 0 & 5 & 3 & 1 & 6 & 4 & 2 \\
6 & 0 & 6 & 5 & 4 & 3 & 2 & 1
\end{tabular}
$$

$$f(x) = 6x^{3}+3x^{2}+4x+1$$
$$g(x) = x^{2}+5x+1$$
\newline
$\begin{array}{lll}
(6x^3 + 3x^2 + 4x + 1)  :  (x^{2}+5x+1)  = 6x+1 \\
\underline{-(6x^3 + 2x^2 +6x)} & &  \\
\qquad x^2 + 5x +1 & & \\
 \ \ \underline{-(x^2 +5x+1)} & &\\
\qquad  \quad r = 0 & &
\end{array} $ \newline \newline
Stąd: $f(x) = (6x+1)g(x) = (6x+1)(x^2 +5x +1) + 0$. \newline
Zatem $NWD(f,g) = g(x) = x^2 +5x+1$

\end{document}