\documentclass{article}
\usepackage[utf8]{inputenc}
\usepackage{polski}

\title{Algebra 2 - Zestaw 10}
\author{Wojciech Szlosek}
\date{April 2020}

\begin{document}

\maketitle

\section{Zadanie 1, (a)}
Niech N będzie zbiorem liczb naturalnych. \newline
$(3N, +)$, gdzie $3N = \{0,3,6,...\} = G$ \newline \newline
Podana struktura jest półgrupą, ponieważ działanie $+$ jest łączne. \newline 
Jest monoidem, ponieważ w zbiorze $G$ istnieje element neutralny $e = 0$, gdzie $x + e = e+ x = x$, dla każdego $x$ należącego do $G$.\newline
Nie jest grupą (a co za tym idzie również grupą abelową, ponieważ nie istnieje dla każdego $x$ należącego do $G$, takie $x^{'}$, że $x + x^{'} = 0 = x^{'} + x$. \newline

\section{Zadanie 1, (b)}
Niech N będzie zbiorem liczb naturalnych. \newline
$(3N, +)$, $G = \{3, 6, 9,...\}$ \newline \newline
Struktura jest półgrupą, ponieważ dodawanie jest łączne. \newline
Nie jedt jednak monoidem (a co za tym idzie nie jest również grupą i grupą abelową), ponieważ nie istnieje element neutralny $e$, który należy do $G$, taki że: $x+e = e+x = e$.

\section{Zadanie 3}
$x*y = x+y - xy$, $x, y$ - liczby całkowite \newline
Mamy udowodnić, że podana struktura jest monoidem, czyli, że jest łączna oraz istnieje dla niej element neutralny. \newline \newline
Widzimy od razu, że jest łączna, bo dodawanie i odejmowanie jest generalnie łączne. \newline
Ponadto istnieje taki element całkowity $e = 0$, taki że: $x+e-xe=e+x-ex=x$. \newline \newline
CND.
\end{document}