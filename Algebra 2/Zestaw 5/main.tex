\documentclass{article}
\usepackage[utf8]{inputenc}

\title{Algebra 2 - Zestaw 5}
\author{Wojciech Szlosek}
\date{March 2020}

\begin{document}

\maketitle

\section{Zadanie 1}

\subsection{a)}

$det \left[\begin{array}{ccc}
3 & 4 & 1 \\
1 & 2 & 99\\
-0.5 & 2 & 3\end{array}\right] + 522 \cdot (\sum_{n=1}^{inf} \frac{1}{(2n-1)(2n+1)} )- 3523 \cdot \lim_{n \to inf} (3n^2 +n)^{\frac{1}{n}}$



\subsection{b)}

Jak wiadomo, przekształcenie liniowe $L: U \to V$, gdzie $U$ i $V$ są przestrzeniami liniowymi skończenie wymiarowymi, jest odwracalne wtw. gdy macierz A jest nieosobliwa. Macierz przekształcenia odwrotnego $L^{-1}$ jest zaś równa macierzy $A^{-1}$. \newline

$L(x,y,z) = (y+2z,x+y+z,2x+3y+2z)$ \\ \newline
$A = \left[\begin{array}{ccc}
0 & 1 & 2 \\
1 & 1 & 1 \\
2 & 3 & 2 \end{array}\right]$
$det A = 2$ \newline \newline 
Skoro jest różny od zera, to przekształcenie rozważane w tym przykładzie jest odwracalne. \newline \newline
$A^{-1} = \left[\begin{array}{ccc}
-\frac{1}{2} & 2 & -\frac{1}{2} \\
0 & -2 & 1 \\
\frac{1}{2} & 1 & -\frac{1}{2} \end{array}\right]$ \newline
$$(Odp.) L^{-1}(x,y,z) = (-\frac{1}{2}x+2y-\frac{1}{2}z,-2y+z,\frac{1}{2}x+y-\frac{1}{2}z)$$

\section{Zadanie 6}

\subsection{a)}

$$A = \left[\begin{array}{ccc}
1 & 0 & 3 \\
0 & 2 & 0 \\
2 & 0 & -1 \end{array}\right]$$

$L^{3}(u_{1}-2u_{2}+u_{3})$ \newline \newline
Macierz przekształcenia $L^{3}$ jest równa $A^{3}$. \newline \newline
$A^{3} = \left[\begin{array}{ccc}
1 & 0 & 3 \\
0 & 2 & 0 \\
2 & 0 & -1 \end{array}\right] \cdot \left[\begin{array}{ccc}
1 & 0 & 3 \\
0 & 2 & 0 \\
2 & 0 & -1 \end{array}\right] \cdot \left[\begin{array}{ccc}
1 & 0 & 3 \\
0 & 2 & 0 \\
2 & 0 & -1 \end{array}\right] = \left[\begin{array}{ccc}
7 & 0 & 0 \\
0 & 4 & 0 \\
0 & 0 & 7 \end{array}\right] \cdot \left[\begin{array}{ccc}
1 & 0 & 3 \\
0 & 2 & 0 \\
2 & 0 & -1 \end{array}\right]$ 
$A^{3} = \left[\begin{array}{ccc}
7 & 0 & 21 \\
0 & 8 & 0 \\
14 & 0 & -7 \end{array}\right]$ \newline \newline
$$A^{3} \cdot \left[\begin{array}{c}
1 \\
-2 \\
1 \end{array}\right] = \left[\begin{array}{ccc}
28 \\
-16\\
7  \end{array}\right]$$ \newline
Więc: $L^{3}(u_{1}-2u_{2}+u_{3}) = 28u_{1}-16u_{2}+7u_{3}$ (odp.)

\end{document}
