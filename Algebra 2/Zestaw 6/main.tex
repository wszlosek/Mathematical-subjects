\documentclass{article}
\usepackage[utf8]{inputenc}

\title{Algebra 2 - Zestaw 6}
\author{Wojciech Szlosek}
\date{March 2020}

\begin{document}

\maketitle

\section{Zadanie 2}

\subsection{a)}

$L(x,y) = (4x+2y,-x+y)$ \newline

Wystarczy wyznaczyć wartości własne i wektory własne macierzy badanych przekształceń liniowych w wybranych bazach przestrzeni liniowych. Posługujemy się bazami standardowymi. \newline

Rozwiązujemy równanie charakterystyczne przekształcenia $L$:
$$det(A - I\lambda) = \left|\begin{array}{cc}4-\lambda&2\\-1&1-\lambda\end{array}\right| = \lambda^{2}-5\lambda + 6 = 0$$
$$\lambda_{1} = 2$$
$$\lambda_{2} = 3$$
Niech $v = (x,y)$ oraz $t = (x,y)$ będą jej wektorami własnymi.
Rozważmy więc dwie "opcje".

\subsubsection{$\lambda_{1} = 2$}

$v = (x,y)$
$$(A-I\lambda)\left[\begin{array}{c}x\\y\end{array}\right] =\left[\begin{array}{cc}2&2\\-1&-1\end{array}\right] \left[\begin{array}{c}x\\y\end{array}\right] = \left[\begin{array}{c}2x+2y\\-x-y\end{array}\right] = \left[\begin{array}{c}0\\0\end{array}\right] $$

$$y = -x$$
$$v = (x,-x)$$ 
$$W_{2} = \{(x,-x)\} = \underline{lin\{(1,-1)\}}$$

\subsubsection{$\lambda_{2} = 3$}

$t = (x,y)$
$$(A-I\lambda)\left[\begin{array}{c}x\\y\end{array}\right] =\left[\begin{array}{cc}1&2\\-1&-2\end{array}\right] \left[\begin{array}{c}x\\y\end{array}\right] = \left[\begin{array}{c}x+2y\\-x-2y\end{array}\right] = \left[\begin{array}{c}0\\0\end{array}\right] $$

$$x = -2y$$
$$v = (-2y,y)$$ 
$$W_{3} = \{(-2y,y)\} = \underline{lin\{(-2,1)\}} $$

\subsection{b)}

$L(x,y) = (2x+y,-x+4y)$ \newline

Wystarczy wyznaczyć wartości własne i wektory własne macierzy badanych przekształceń liniowych w wybranych bazach przestrzeni liniowych. Posługujemy się bazami standardowymi. \newline

Rozwiązujemy równanie charakterystyczne przekształcenia $L$:
$$det(A - I\lambda) = \left|\begin{array}{cc}2-\lambda&1\\-1&4-\lambda\end{array}\right| = (2-\lambda)(4-\lambda)+1 = (\lambda-3)^{2} = 0$$

Równanie to ma jedyną wartość własną $\lambda = 3$. Niech $v = (x,y)$ będzie jej wektorem własnym.

$$(A-I\lambda)\left[\begin{array}{c}x\\y\end{array}\right] = \left[\begin{array}{cc}-1&1\\-1&1\end{array}\right]\left[\begin{array}{c}x\\y\end{array}\right] = \left[\begin{array}{c}-x+y\\-x+y\end{array}\right] = \left[\begin{array}{c}0\\0\end{array}\right]$$
Stąd $v = (y,y)$ (bo $x=y$). \newline Ponadto przestrzeń wektorów własnych równa się: $W = {(y,y)} = lin\{(1,1)\}$


\section{Zadanie 3}

\subsection{a)}

$$L(x,y) = (3x-y,10x-3y)$$
$$det(A-I\lambda) = det \left[\begin{array}{cc}3-\lambda&-1\\10&-3-\lambda\end{array}\right] = -(3-\lambda)(3+\lambda)+10 = \lambda^{2} + 1 = 0$$
$$\lambda_{1} = i, \lambda{2} = -i :<wartosci wlasne>$$

$$(A-I\lambda_{1})\left[\begin{array}{c}x\\y\end{array}\right] = \left[\begin{array}{cc}3-i&-1\\10&-3-i\end{array}\right]\left[\begin{array}{c}x\\y\end{array}\right] = 0 \Rightarrow y = x(3-i)$$

$$(A-I\lambda_{2})\left[\begin{array}{c}x\\y\end{array}\right] = \left[\begin{array}{cc}3+i&-1\\10&i-3\end{array}\right]\left[\begin{array}{c}x\\y\end{array}\right] = 0 \Rightarrow y = x(i+3)$$

$$W_{i} = (x,x(3-i)) = lin\{(1,3-i)\} $$
$$W_{-i} = (x,x(3+i)) = lin\{(1,3+i)\} $$

\end{document}