\documentclass{article}
\usepackage[utf8]{inputenc}
\usepackage{polski}
\usepackage{amsmath}

\title{Algebra 2 - Zestaw 11}
\author{Wojciech Szlosek}
\date{May 2020}

\begin{document}

\maketitle

\section{Zadanie 1, (a)}

Do danej w treści zadania permutacji $\delta$, uzupełnijmy $r$. \newline
$$r = (5,8,1,4,6)$$

$$ \[
  r = \bigl(\begin{smallmatrix}
    1 & 2 & 3 & 4&5&6&7&8&9&10\\
    4 & 2 & 3 & 6&8&5&7&1&9&10
  \end{smallmatrix}\bigr)
\] $$

Teraz możemy przejść do odpowiedzi na punkty z zadania.

$$ \[
  \delta ^{-1} = \bigl(\begin{smallmatrix}
    4 & 7 & 5 & 3&8&6&10&1&9&2\\
    1 & 2 & 3 & 4&5&6&7&8&9&10
  \end{smallmatrix}\bigr)
\] $$

$$ \[
  r ^{-1} = \bigl(\begin{smallmatrix}
    4 & 2 & 3 & 6&8&5&7&1&9&10\\
    1 & 2 & 3 & 4&5&6&7&8&9&10
  \end{smallmatrix}\bigr)
\] $$ \newline

Dla $i = \{ 1, ..., 10 \}$, stwórzmy $\delta(r(i))$, a na tej podstawie $\delta r$. Dla kolejnych wartości $i$ mamy kolejne wartości $\delta(r(i))$ równe: 3, 7, 5, 6, 1, 8, 10, 4, 9, 2. Zapiszmy więc:

$$ \[
  \delta r = \bigl(\begin{smallmatrix}
    1 &2 & 3 & 4&5&6&7&8&9&10\\
    3 & 7 & 5 & 6&1&8&10&4&9&2
  \end{smallmatrix}\bigr)
\] $$ 
Dla $r \delta$ postępujemy analogicznie, ostatecznie mamy więc:

$$ \[
  r \delta = \bigl(\begin{smallmatrix}
    1 &2 & 3 & 4&5&6&7&8&9&10\\
    6 & 7 & 8 & 3&1&5&10&4&9&2
  \end{smallmatrix}\bigr)
\] $$ 


$$ \[
  (\delta r)^{-1} = \bigl(\begin{smallmatrix}
    3 & 7 & 5 & 6&1&8&10&4&9&2\\
    1 & 2 & 3 & 4&5&6&7&8&9&10
  \end{smallmatrix}\bigr)
\] $$

$$ \[
  (r \delta)^{-1} = \bigl(\begin{smallmatrix}
    6 & 7 & 8 & 3&1&5&10&4&9&2\\
    1 & 2 & 3 & 4&5&6&7&8&9&10
  \end{smallmatrix}\bigr)
\] $$


\section{Zadanie 1, (d)}

Niech $o$ oznacza złożenie.
$$\delta = (1,4,3,5,8) o (10,2,7) = (1,8) o (1,5)o(1,3)o(1,4)o(10,7)o(10,2)$$
Parzysta liczba transpozycji, czyli 
$sgn(\delta) = 1$. \newline

$$r = (1,4,6,5,8) = (1,8)o(1,5)o(1,6)o(1,4)$$
Parzysta liczba transpozycji, czyli 
$sgn(r) = 1$.

\end{document}